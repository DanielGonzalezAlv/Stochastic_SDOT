\documentclass[
     12pt,         % font size
     a4paper,      % paper format
     BCOR=10mm,     % binding correction
     DIV=14,        % stripe size for margin calculation
%     liststotoc,   % table listing in toc
%     bibtotoc,     % bibliography in toc
%     idxtotoc,     % index in toc
%     parskip       % paragraph skip instad of paragraph indent
     ]{scrreprt}

% Preamble
%\documentclass[
%     12pt,         % font size
%     a4paper,      % paper format
%     BCOR10mm,     % binding correction
%     DIV14,        % stripe size for margin calculation
%%     liststotoc,   % table listing in toc
%%     bibtotoc,     % bibliography in toc
%%     idxtotoc,     % index in toc
%%     parskip       % paragraph skip instad of paragraph indent
%     ]{scrreprt}
%
%%%%%%%%%%%%%%%%%%%%%%%%%%%%%%%%%%%%%%%%%%%%%%%%%%%%%%%%%%%%

% PACKAGES:

% Language :
\usepackage[ngerman]{babel}
% Input and font encoding
\usepackage[utf8]{inputenc}
\usepackage[T1]{fontenc}

%% Font
%\setmainfont{Linux Libertine O}
%\setsansfont{Linux Biolinum O}

% Index-generation
\usepackage{makeidx}

% Einbinden von URLs:
\usepackage{url}
% Special \LaTex symbols (e.g. \BibTeX):
%\usepackage{doc}
% Include Graphic-files:
\usepackage{graphicx}
% Include doc++ generated tex-files:
%\usepackage{docxx}
% Include PDF links
%\usepackage[pdftex, bookmarks=true]{hyperref}

% Fuer anderthalbzeiligen Textsatz
\usepackage{setspace}

% hyperrefs in the documents
\usepackage[bookmarks=true,colorlinks,pdfpagelabels,pdfstartview = FitH,bookmarksopen = true,bookmarksnumbered = true,linkcolor = black,plainpages = false,hypertexnames = false,citecolor = black,urlcolor=black]{hyperref} 
%\usepackage{hyperref}

%%Hyphenation - Silbentrennung
%\hyphenation{Prim-ideal}

%%Bibliography
%\usepackage[backend=biber,style=alphabetic]{biblatex}
%\addbibresource{mybib.bib}
%\ExecuteBibliographyOptions{%
% isbn=false, doi=false,url=false,eprint=false%
%}%
%\DeclareSourcemap{
%  \maps[datatype=bibtex, overwrite]{
%    \map{
%      \step[fieldset=language, null]
%      \step[fieldset=note, null]
%      \step[fieldset=pagetotal, null]
%    }
%  }
%}

%Math packages
\usepackage{mathtools}%mathtools lädt amsmath
\usepackage{amssymb}
\usepackage{amsthm}

%% Use the Libertine font family for math as well
%\usepackage[libertine]{newtxmath}
%
%%Math operators
\DeclareMathOperator{\N}{\mathbb{N}}
\DeclareMathOperator{\Z}{\mathbb{Z}}
\DeclareMathOperator{\Q}{\mathbb{Q}}
\DeclareMathOperator{\R}{\mathbb{R}}
\DeclareMathOperator{\C}{\mathbb{C}}
\DeclareMathOperator{\F}{\mathbb{F}}
\DeclareMathOperator{\GL}{GL}


%%
%%PAGE REFORMATTING
%%
%\setlength{\oddsidemargin}{1.5cm}
%\setlength{\evensidemargin}{0cm}
%\setlength{\topmargin}{1mm}
%\setlength{\headheight}{1.36cm}
%\setlength{\headsep}{1.00cm}
%\setlength{\textheight}{19cm}
%\setlength{\textwidth}{14.5cm}
%\setlength{\marginparsep}{1mm}
%\setlength{\marginparwidth}{3cm}
%\setlength{\footskip}{2.36cm}

%% TODO command
%\usepackage[colorinlistoftodos,prependcaption,textsize=tiny]{todonotes}
%\newcommandx{\unsure}[2][1=]{\todo[linecolor=red,backgroundcolor=red!25,bordercolor=red,#1]{#2}}
%\newcommandx{\change}[2][1=]{\todo[linecolor=blue,backgroundcolor=blue!25,bordercolor=blue,#1]{#2}}
%\newcommandx{\info}[2][1=]{\todo[linecolor=OliveGreen,backgroundcolor=OliveGreen!25,bordercolor=OliveGreen,#1]{#2}}
%\newcommandx{\improvement}[2][1=]{\todo[linecolor=Plum,backgroundcolor=Plum!25,bordercolor=Plum,#1]{#2}}
%\newcommandx{\thiswillnotshow}[2][1=]{\todo[disable,#1]{#2}}


%%%%%%%%%%%%%%%%%%%%%%%%%%%%%%%%%%%%%%%%%%%%%%%%%%%%%%%%%%%%

% OTHER SETTINGS:

% Pagestyle:
\pagestyle{headings}

% Choose language
\newcommand{\setlang}[1]{\selectlanguage{#1}\nonfrenchspacing}


%% Name for the Table of contents
%\renewcommand{\contentsname}{Whatever}

% Avoid Problems by including other files
\usepackage{standalone}

%%%%%%%%%%%%%%%%%%%%%%%%%%%%%%%%%%%%%%%%%%%%%%%%%%%%%%%%%%%%%

\begin{document}

% Page numbering
\pagenumbering{roman} 

%
\documentclass[
     12pt,         % font size
     a4paper,      % paper format
     BCOR10mm,     % binding correction
     DIV14,        % stripe size for margin calculation
%     liststotoc,   % table listing in toc
%     bibtotoc,     % bibliography in toc
%     idxtotoc,     % index in toc
%     parskip       % paragraph skip instad of paragraph indent
     ]{scrreprt}

%\documentclass[
%     12pt,         % font size
%     a4paper,      % paper format
%     BCOR10mm,     % binding correction
%     DIV14,        % stripe size for margin calculation
%%     liststotoc,   % table listing in toc
%%     bibtotoc,     % bibliography in toc
%%     idxtotoc,     % index in toc
%%     parskip       % paragraph skip instad of paragraph indent
%     ]{scrreprt}
%
%%%%%%%%%%%%%%%%%%%%%%%%%%%%%%%%%%%%%%%%%%%%%%%%%%%%%%%%%%%%

% PACKAGES:

% Language :
\usepackage[ngerman]{babel}
% Input and font encoding
\usepackage[utf8]{inputenc}
\usepackage[T1]{fontenc}

%% Font
%\setmainfont{Linux Libertine O}
%\setsansfont{Linux Biolinum O}

% Index-generation
\usepackage{makeidx}

% Einbinden von URLs:
\usepackage{url}
% Special \LaTex symbols (e.g. \BibTeX):
%\usepackage{doc}
% Include Graphic-files:
\usepackage{graphicx}
% Include doc++ generated tex-files:
%\usepackage{docxx}
% Include PDF links
%\usepackage[pdftex, bookmarks=true]{hyperref}

% Fuer anderthalbzeiligen Textsatz
\usepackage{setspace}

% hyperrefs in the documents
\usepackage[bookmarks=true,colorlinks,pdfpagelabels,pdfstartview = FitH,bookmarksopen = true,bookmarksnumbered = true,linkcolor = black,plainpages = false,hypertexnames = false,citecolor = black,urlcolor=black]{hyperref} 
%\usepackage{hyperref}

%%Hyphenation - Silbentrennung
%\hyphenation{Prim-ideal}

%%Bibliography
%\usepackage[backend=biber,style=alphabetic]{biblatex}
%\addbibresource{mybib.bib}
%\ExecuteBibliographyOptions{%
% isbn=false, doi=false,url=false,eprint=false%
%}%
%\DeclareSourcemap{
%  \maps[datatype=bibtex, overwrite]{
%    \map{
%      \step[fieldset=language, null]
%      \step[fieldset=note, null]
%      \step[fieldset=pagetotal, null]
%    }
%  }
%}

%Math packages
\usepackage{mathtools}%mathtools lädt amsmath
\usepackage{amssymb}
\usepackage{amsthm}

%% Use the Libertine font family for math as well
%\usepackage[libertine]{newtxmath}
%
%%Math operators
\DeclareMathOperator{\N}{\mathbb{N}}
\DeclareMathOperator{\Z}{\mathbb{Z}}
\DeclareMathOperator{\Q}{\mathbb{Q}}
\DeclareMathOperator{\R}{\mathbb{R}}
\DeclareMathOperator{\C}{\mathbb{C}}
\DeclareMathOperator{\F}{\mathbb{F}}
\DeclareMathOperator{\GL}{GL}


%%
%%PAGE REFORMATTING
%%
%\setlength{\oddsidemargin}{1.5cm}
%\setlength{\evensidemargin}{0cm}
%\setlength{\topmargin}{1mm}
%\setlength{\headheight}{1.36cm}
%\setlength{\headsep}{1.00cm}
%\setlength{\textheight}{19cm}
%\setlength{\textwidth}{14.5cm}
%\setlength{\marginparsep}{1mm}
%\setlength{\marginparwidth}{3cm}
%\setlength{\footskip}{2.36cm}

%% TODO command
%\usepackage[colorinlistoftodos,prependcaption,textsize=tiny]{todonotes}
%\newcommandx{\unsure}[2][1=]{\todo[linecolor=red,backgroundcolor=red!25,bordercolor=red,#1]{#2}}
%\newcommandx{\change}[2][1=]{\todo[linecolor=blue,backgroundcolor=blue!25,bordercolor=blue,#1]{#2}}
%\newcommandx{\info}[2][1=]{\todo[linecolor=OliveGreen,backgroundcolor=OliveGreen!25,bordercolor=OliveGreen,#1]{#2}}
%\newcommandx{\improvement}[2][1=]{\todo[linecolor=Plum,backgroundcolor=Plum!25,bordercolor=Plum,#1]{#2}}
%\newcommandx{\thiswillnotshow}[2][1=]{\todo[disable,#1]{#2}}


%%%%%%%%%%%%%%%%%%%%%%%%%%%%%%%%%%%%%%%%%%%%%%%%%%%%%%%%%%%%

% OTHER SETTINGS:

% Pagestyle:
\pagestyle{headings}

% Choose language
\newcommand{\setlang}[1]{\selectlanguage{#1}\nonfrenchspacing}


%%%%%%%%%%%%%%%%%%%%%%%%%%%%%%%%%%%%%%%%%%%%%%%%%%%%%%%%%%%%

\begin{document}

\begin{titlepage}


\vspace*{1cm}
\begin{center}
\vspace*{3cm}
\textbf{ 
\Large University of Heidelberg \\
\smallskip
\Large Department of Mathematics and Computer Science\\
\smallskip
\Large Image \& Pattern Analysis Group \\
\smallskip
}

\vspace{3cm}

\textbf{\large Master-Thesis} 

\vspace{0.5\baselineskip}
{\huge
\textbf{Semi-discrete Optimal Transport}
}
\end{center}

\vfill 

{\large
\begin{tabular}[l]{ll}
Name: & Daniel Gonzalez\\
Enrolment number: & 3112012\\
Supervisor: & Prof. Christoph Schnörr \\
Date: & \today
\end{tabular}
}

\end{titlepage}

\end{document}

%
\onehalfspacing
\thispagestyle{empty}
\vspace*{100pt}

%
\documentclass[
     12pt,         % font size
     a4paper,      % paper format
     BCOR10mm,     % binding correction
     DIV14,        % stripe size for margin calculation
%     liststotoc,   % table listing in toc
%     bibtotoc,     % bibliography in toc
%     idxtotoc,     % index in toc
%     parskip       % paragraph skip instad of paragraph indent
     ]{scrreprt}

%\documentclass[
%     12pt,         % font size
%     a4paper,      % paper format
%     BCOR10mm,     % binding correction
%     DIV14,        % stripe size for margin calculation
%%     liststotoc,   % table listing in toc
%%     bibtotoc,     % bibliography in toc
%%     idxtotoc,     % index in toc
%%     parskip       % paragraph skip instad of paragraph indent
%     ]{scrreprt}
%
%%%%%%%%%%%%%%%%%%%%%%%%%%%%%%%%%%%%%%%%%%%%%%%%%%%%%%%%%%%%

% PACKAGES:

% Language :
\usepackage[ngerman]{babel}
% Input and font encoding
\usepackage[utf8]{inputenc}
\usepackage[T1]{fontenc}

%% Font
%\setmainfont{Linux Libertine O}
%\setsansfont{Linux Biolinum O}

% Index-generation
\usepackage{makeidx}

% Einbinden von URLs:
\usepackage{url}
% Special \LaTex symbols (e.g. \BibTeX):
%\usepackage{doc}
% Include Graphic-files:
\usepackage{graphicx}
% Include doc++ generated tex-files:
%\usepackage{docxx}
% Include PDF links
%\usepackage[pdftex, bookmarks=true]{hyperref}

% Fuer anderthalbzeiligen Textsatz
\usepackage{setspace}

% hyperrefs in the documents
\usepackage[bookmarks=true,colorlinks,pdfpagelabels,pdfstartview = FitH,bookmarksopen = true,bookmarksnumbered = true,linkcolor = black,plainpages = false,hypertexnames = false,citecolor = black,urlcolor=black]{hyperref} 
%\usepackage{hyperref}

%%Hyphenation - Silbentrennung
%\hyphenation{Prim-ideal}

%%Bibliography
%\usepackage[backend=biber,style=alphabetic]{biblatex}
%\addbibresource{mybib.bib}
%\ExecuteBibliographyOptions{%
% isbn=false, doi=false,url=false,eprint=false%
%}%
%\DeclareSourcemap{
%  \maps[datatype=bibtex, overwrite]{
%    \map{
%      \step[fieldset=language, null]
%      \step[fieldset=note, null]
%      \step[fieldset=pagetotal, null]
%    }
%  }
%}

%Math packages
\usepackage{mathtools}%mathtools lädt amsmath
\usepackage{amssymb}
\usepackage{amsthm}

%% Use the Libertine font family for math as well
%\usepackage[libertine]{newtxmath}
%
%%Math operators
\DeclareMathOperator{\N}{\mathbb{N}}
\DeclareMathOperator{\Z}{\mathbb{Z}}
\DeclareMathOperator{\Q}{\mathbb{Q}}
\DeclareMathOperator{\R}{\mathbb{R}}
\DeclareMathOperator{\C}{\mathbb{C}}
\DeclareMathOperator{\F}{\mathbb{F}}
\DeclareMathOperator{\GL}{GL}


%%
%%PAGE REFORMATTING
%%
%\setlength{\oddsidemargin}{1.5cm}
%\setlength{\evensidemargin}{0cm}
%\setlength{\topmargin}{1mm}
%\setlength{\headheight}{1.36cm}
%\setlength{\headsep}{1.00cm}
%\setlength{\textheight}{19cm}
%\setlength{\textwidth}{14.5cm}
%\setlength{\marginparsep}{1mm}
%\setlength{\marginparwidth}{3cm}
%\setlength{\footskip}{2.36cm}

%% TODO command
%\usepackage[colorinlistoftodos,prependcaption,textsize=tiny]{todonotes}
%\newcommandx{\unsure}[2][1=]{\todo[linecolor=red,backgroundcolor=red!25,bordercolor=red,#1]{#2}}
%\newcommandx{\change}[2][1=]{\todo[linecolor=blue,backgroundcolor=blue!25,bordercolor=blue,#1]{#2}}
%\newcommandx{\info}[2][1=]{\todo[linecolor=OliveGreen,backgroundcolor=OliveGreen!25,bordercolor=OliveGreen,#1]{#2}}
%\newcommandx{\improvement}[2][1=]{\todo[linecolor=Plum,backgroundcolor=Plum!25,bordercolor=Plum,#1]{#2}}
%\newcommandx{\thiswillnotshow}[2][1=]{\todo[disable,#1]{#2}}


%%%%%%%%%%%%%%%%%%%%%%%%%%%%%%%%%%%%%%%%%%%%%%%%%%%%%%%%%%%%

% OTHER SETTINGS:

% Pagestyle:
\pagestyle{headings}

% Choose language
\newcommand{\setlang}[1]{\selectlanguage{#1}\nonfrenchspacing}


%%%%%%%%%%%%%%%%%%%%%%%%%%%%%%%%%%%%%%%%%%%%%%%%%%%%%%%%%%%%

\begin{document}
\noindent
Ich versichere, dass ich diese Master-Arbeit selbststndig verfasst und nur die angegebenen
Quellen und Hilfsmittel verwendet habe und die Grundstze und
Empfehlungen ``Verantwortung in der Wissenschaft'' der Universitt Heidelberg beachtet wurden. 

\vspace*{50pt}
\noindent

\underline{\phantom{mmmmmmmmmmmmmmmmmmmm}}

\medskip
\noindent 
Abgabedatum: \today
\end{document}


%
\newpage

%
\chapter*{Zusammenfassung}

\documentclass[
     12pt,         % font size
     a4paper,      % paper format
     BCOR10mm,     % binding correction
     DIV14,        % stripe size for margin calculation
%     liststotoc,   % table listing in toc
%     bibtotoc,     % bibliography in toc
%     idxtotoc,     % index in toc
%     parskip       % paragraph skip instad of paragraph indent
     ]{scrreprt}

%\documentclass[
%     12pt,         % font size
%     a4paper,      % paper format
%     BCOR10mm,     % binding correction
%     DIV14,        % stripe size for margin calculation
%%     liststotoc,   % table listing in toc
%%     bibtotoc,     % bibliography in toc
%%     idxtotoc,     % index in toc
%%     parskip       % paragraph skip instad of paragraph indent
%     ]{scrreprt}
%
%%%%%%%%%%%%%%%%%%%%%%%%%%%%%%%%%%%%%%%%%%%%%%%%%%%%%%%%%%%%

% PACKAGES:

% Language :
\usepackage[ngerman]{babel}
% Input and font encoding
\usepackage[utf8]{inputenc}
\usepackage[T1]{fontenc}

%% Font
%\setmainfont{Linux Libertine O}
%\setsansfont{Linux Biolinum O}

% Index-generation
\usepackage{makeidx}

% Einbinden von URLs:
\usepackage{url}
% Special \LaTex symbols (e.g. \BibTeX):
%\usepackage{doc}
% Include Graphic-files:
\usepackage{graphicx}
% Include doc++ generated tex-files:
%\usepackage{docxx}
% Include PDF links
%\usepackage[pdftex, bookmarks=true]{hyperref}

% Fuer anderthalbzeiligen Textsatz
\usepackage{setspace}

% hyperrefs in the documents
\usepackage[bookmarks=true,colorlinks,pdfpagelabels,pdfstartview = FitH,bookmarksopen = true,bookmarksnumbered = true,linkcolor = black,plainpages = false,hypertexnames = false,citecolor = black,urlcolor=black]{hyperref} 
%\usepackage{hyperref}

%%Hyphenation - Silbentrennung
%\hyphenation{Prim-ideal}

%%Bibliography
%\usepackage[backend=biber,style=alphabetic]{biblatex}
%\addbibresource{mybib.bib}
%\ExecuteBibliographyOptions{%
% isbn=false, doi=false,url=false,eprint=false%
%}%
%\DeclareSourcemap{
%  \maps[datatype=bibtex, overwrite]{
%    \map{
%      \step[fieldset=language, null]
%      \step[fieldset=note, null]
%      \step[fieldset=pagetotal, null]
%    }
%  }
%}

%Math packages
\usepackage{mathtools}%mathtools lädt amsmath
\usepackage{amssymb}
\usepackage{amsthm}

%% Use the Libertine font family for math as well
%\usepackage[libertine]{newtxmath}
%
%%Math operators
\DeclareMathOperator{\N}{\mathbb{N}}
\DeclareMathOperator{\Z}{\mathbb{Z}}
\DeclareMathOperator{\Q}{\mathbb{Q}}
\DeclareMathOperator{\R}{\mathbb{R}}
\DeclareMathOperator{\C}{\mathbb{C}}
\DeclareMathOperator{\F}{\mathbb{F}}
\DeclareMathOperator{\GL}{GL}


%%
%%PAGE REFORMATTING
%%
%\setlength{\oddsidemargin}{1.5cm}
%\setlength{\evensidemargin}{0cm}
%\setlength{\topmargin}{1mm}
%\setlength{\headheight}{1.36cm}
%\setlength{\headsep}{1.00cm}
%\setlength{\textheight}{19cm}
%\setlength{\textwidth}{14.5cm}
%\setlength{\marginparsep}{1mm}
%\setlength{\marginparwidth}{3cm}
%\setlength{\footskip}{2.36cm}

%% TODO command
%\usepackage[colorinlistoftodos,prependcaption,textsize=tiny]{todonotes}
%\newcommandx{\unsure}[2][1=]{\todo[linecolor=red,backgroundcolor=red!25,bordercolor=red,#1]{#2}}
%\newcommandx{\change}[2][1=]{\todo[linecolor=blue,backgroundcolor=blue!25,bordercolor=blue,#1]{#2}}
%\newcommandx{\info}[2][1=]{\todo[linecolor=OliveGreen,backgroundcolor=OliveGreen!25,bordercolor=OliveGreen,#1]{#2}}
%\newcommandx{\improvement}[2][1=]{\todo[linecolor=Plum,backgroundcolor=Plum!25,bordercolor=Plum,#1]{#2}}
%\newcommandx{\thiswillnotshow}[2][1=]{\todo[disable,#1]{#2}}


%%%%%%%%%%%%%%%%%%%%%%%%%%%%%%%%%%%%%%%%%%%%%%%%%%%%%%%%%%%%

% OTHER SETTINGS:

% Pagestyle:
\pagestyle{headings}

% Choose language
\newcommand{\setlang}[1]{\selectlanguage{#1}\nonfrenchspacing}


%%%%%%%%%%%%%%%%%%%%%%%%%%%%%%%%%%%%%%%%%%%%%%%%%%%%%%%%%%%%

\begin{document}

% This file contains the German version of your abstract, with about 300-500 words

Die Zusammenfassung muss auf Deutsch \textbf{und} auf Englisch geschrieben
werden. Die Zusammenfassung sollte zwischen einer halben und einer
ganzen Seite lang sein. Sie soll den Kontext der Arbeit, die
Problemstellung, die Zielsetzung und die entwickelten Methoden sowie
Erkenntnisse bzw.~Ergebnisse bersichtlich und verstndlich
beschreiben.

\end{document}


%
\newpage

%
\chapter*{Abstract}
% !TeX root = main.tex

\begin{abstract}
Diese Zusammenfassung der Abschlussarbeit ist auf Deutsch und wird korrekt getrennt.

\blindtext[1]
\end{abstract}
\begin{otherlanguage}{english}
\begin{abstract}
This abstract of the thesis is in English and the syllables are divided correctly.

\blindtext[1]
\end{abstract}
\end{otherlanguage}

%
\newpage


% Table of contents (Inhaltsverzeichnis)
\tableofcontents
\cleardoublepage

%
\pagenumbering{arabic} 

% List of figures (Abbildungsverzeichnis):
%\listoffigures
% List of tables (Tabellenverzeichnis):
%\listoftables


%%%%%%%%%%%%%%%%%%%%%%%%%%%%%%%%%%%%%%%%%%%%%%%%%%%%%%%%%%%%%%%
%
%\chapter{Einleitung}\label{intro}
%
%Dieses Kapitel gibt einen berblick ber die Arbeit. Gerade der
%Abschnitt zur Motivation soll allgemein verstndlich geschrieben
%werden. Die Einleitung sollte auch wichtige Referenzen enthalten. 
%
%\section{Motivation}
%
%Worum geht es? Beispiel(e)! Illustrationen sind hier meist sinnvoll
%zum Verstndnis. Warum ist das Thema wichtig? In welchem Kontext?
%
%\section{Ziele der Arbeit}
%
%In diesem Abschnitt sollen neben den Herausforderungen und der
%Problemstellung insbesondere die Ziele der Arbeit beschrieben werden. 
%
%\section{Aufbau der Arbeit}
%
%Dieser Abschnitt wird meist recht kurz gehalten und beschreibt im
%Prinzip nur den Aufbau des Rests der Arbeit. Zum Beispiel: In Kapitel
%\ref{chap:grundlagen} geben wir einen r die  Grundlagen
%zu der Arbeit sowber verwandte Arbeiten. In Kapitel
%\ref{chap:main} stellen wir dann \ldots vor. \ldots etc.
%
%
%\newpage
%
%\chapter{Grundlagen und verwandte Arbeiten}
%\label{chap:grundlagen}
%
%Die ersten paar Abschnitte in diesem Kapitelhren in die Grundlagen
%zur Arbeit ein. Das knnen beispielsweise Grundlagen zu Netzwerken
%oder zur Informationsextraktion sein. 
%
%\section{(Beispiel) Netzwerke}
%\label{sec:networks}
%
%\section{(Beispiel) Informationsextraktion}
%\label{sec:ie}
%
%\section{Verwandte Arbeiten}
%\label{sec:related}
%
%Typischerweise im letzten Abschnitt dieses Kapitels wird dann auf
%verwandte Arbeiten eingegangen. Entsprechende Arbeiten sind geeignet
%zu zitieren. Beispiel: Die wurde erstmalig in den Arbeiten von Spitz
%und Gertz \cite{Spitz2016a} gezeigt \ldots Details dazu werden in
%dem Buch von Newman zu Netzwerken \cite{Newman2010} erlutert \ldots.
%
%\newpage
%
% Alternative: put content in separate files
% Check the difference between including these files using \input{filename} and \include{filename} and see which one you like better
%\chapter{Einleitung}\label{intro}
%% !TeX root = ../main.tex

\enquote{Als Gregor Samsa eines Morgens aus unruhigen Träumen erwachte, fand er sich in seinem Bett zu einem ungeheueren Ungeziefer verwandelt. Er lag auf seinem panzerartig harten Rücken und sah, wenn er den Kopf ein wenig hob, seinen gewölbten, braunen, von bogenförmigen Versteifungen geteilten Bauch, auf dessen Höhe sich die Bettdecke, zum gänzlichen Niedergleiten bereit, kaum noch erhalten konnte. Seine vielen, im Vergleich zu seinem sonstigen Umfang kläglich dünnen Beine flimmerten ihm hilflos vor den Augen.} Mit diesen Worten beginnt nicht nur diese Arbeit, sondern auch Franz Kafkas berühmtes Werk \enquote{Die Verwandlung} \cite[vgl.][]{kafka1915}.
\blindtext[2]
%
%\chapter{Voraussetzungen}\label{bg}
%\input{background}

%%%%%%%%%%%%%%%%%%%%%%%%%%%%%%%%%%%%%%%%%%%%%%%%%%%%%%%%%%%%

% Nomenclature
\unsure{ Probably not necessary - Introduce nomenclature and conventions in the text}
\documentclass[
     12pt,         % font size
     a4paper,      % paper format
     BCOR=10mm,     % binding correction
     DIV=14,        % stripe size for margin calculation
%     liststotoc,   % table listing in toc
%     bibtotoc,     % bibliography in toc
%     idxtotoc,     % index in toc
%     parskip       % paragraph skip instad of paragraph indent
     ]{scrreprt}

%\documentclass[
%     12pt,         % font size
%     a4paper,      % paper format
%     BCOR10mm,     % binding correction
%     DIV14,        % stripe size for margin calculation
%%     liststotoc,   % table listing in toc
%%     bibtotoc,     % bibliography in toc
%%     idxtotoc,     % index in toc
%%     parskip       % paragraph skip instad of paragraph indent
%     ]{scrreprt}
%
%%%%%%%%%%%%%%%%%%%%%%%%%%%%%%%%%%%%%%%%%%%%%%%%%%%%%%%%%%%%

% PACKAGES:

% Language :
\usepackage[ngerman]{babel}
% Input and font encoding
\usepackage[utf8]{inputenc}
\usepackage[T1]{fontenc}

%% Font
%\setmainfont{Linux Libertine O}
%\setsansfont{Linux Biolinum O}

% Index-generation
\usepackage{makeidx}

% Einbinden von URLs:
\usepackage{url}
% Special \LaTex symbols (e.g. \BibTeX):
%\usepackage{doc}
% Include Graphic-files:
\usepackage{graphicx}
% Include doc++ generated tex-files:
%\usepackage{docxx}
% Include PDF links
%\usepackage[pdftex, bookmarks=true]{hyperref}

% Fuer anderthalbzeiligen Textsatz
\usepackage{setspace}

% hyperrefs in the documents
\usepackage[bookmarks=true,colorlinks,pdfpagelabels,pdfstartview = FitH,bookmarksopen = true,bookmarksnumbered = true,linkcolor = black,plainpages = false,hypertexnames = false,citecolor = black,urlcolor=black]{hyperref} 
%\usepackage{hyperref}

%%Hyphenation - Silbentrennung
%\hyphenation{Prim-ideal}

%%Bibliography
%\usepackage[backend=biber,style=alphabetic]{biblatex}
%\addbibresource{mybib.bib}
%\ExecuteBibliographyOptions{%
% isbn=false, doi=false,url=false,eprint=false%
%}%
%\DeclareSourcemap{
%  \maps[datatype=bibtex, overwrite]{
%    \map{
%      \step[fieldset=language, null]
%      \step[fieldset=note, null]
%      \step[fieldset=pagetotal, null]
%    }
%  }
%}

%Math packages
\usepackage{mathtools}%mathtools lädt amsmath
\usepackage{amssymb}
\usepackage{amsthm}

%% Use the Libertine font family for math as well
%\usepackage[libertine]{newtxmath}
%
%%Math operators
\DeclareMathOperator{\N}{\mathbb{N}}
\DeclareMathOperator{\Z}{\mathbb{Z}}
\DeclareMathOperator{\Q}{\mathbb{Q}}
\DeclareMathOperator{\R}{\mathbb{R}}
\DeclareMathOperator{\C}{\mathbb{C}}
\DeclareMathOperator{\F}{\mathbb{F}}
\DeclareMathOperator{\GL}{GL}


%%
%%PAGE REFORMATTING
%%
%\setlength{\oddsidemargin}{1.5cm}
%\setlength{\evensidemargin}{0cm}
%\setlength{\topmargin}{1mm}
%\setlength{\headheight}{1.36cm}
%\setlength{\headsep}{1.00cm}
%\setlength{\textheight}{19cm}
%\setlength{\textwidth}{14.5cm}
%\setlength{\marginparsep}{1mm}
%\setlength{\marginparwidth}{3cm}
%\setlength{\footskip}{2.36cm}

%% TODO command
%\usepackage[colorinlistoftodos,prependcaption,textsize=tiny]{todonotes}
%\newcommandx{\unsure}[2][1=]{\todo[linecolor=red,backgroundcolor=red!25,bordercolor=red,#1]{#2}}
%\newcommandx{\change}[2][1=]{\todo[linecolor=blue,backgroundcolor=blue!25,bordercolor=blue,#1]{#2}}
%\newcommandx{\info}[2][1=]{\todo[linecolor=OliveGreen,backgroundcolor=OliveGreen!25,bordercolor=OliveGreen,#1]{#2}}
%\newcommandx{\improvement}[2][1=]{\todo[linecolor=Plum,backgroundcolor=Plum!25,bordercolor=Plum,#1]{#2}}
%\newcommandx{\thiswillnotshow}[2][1=]{\todo[disable,#1]{#2}}


%%%%%%%%%%%%%%%%%%%%%%%%%%%%%%%%%%%%%%%%%%%%%%%%%%%%%%%%%%%%

% OTHER SETTINGS:

% Pagestyle:
\pagestyle{headings}

% Choose language
\newcommand{\setlang}[1]{\selectlanguage{#1}\nonfrenchspacing}


%%%%%%%%%%%%%%%%%%%%%%%%%%%%%%%%%%%%%%%%%%%%%%%%%%%%%%%%%%%%

\begin{document}
   \begin{Large}
       \noindent \textbf{ Nomenclature }
       \hspace{2cm}
   \end{Large}

   \begin{flalign*}
        &\| \cdot \|&                       &\text{Euclidean Norm on } \R^d&\\
        &\langle \cdot, \cdot \rangle&      &\text{Scalar product on } \R^d&\\
        &\R_{\ge 0}&                        &\text{Positive real numbers}&\\
        &\lambda^d&                         &\text{Lebesgue measure on} \R^d&\\
        &\B^d &                             &\text{Borel} \sigma\text{-algebra on } \R^d&\\
        &\Pot(\cdot)&                       &\text{Power set of a set}&\\
        &|\cdot|&                           &\text{Cardinality of a set}&\\
        &\LL(\mu)&                          &\mu \text{-integrable functions}&\\
    \end{flalign*}


\end{document}
 
%\newpage


% Genearal equations / Motivation
\documentclass[
     12pt,         % font size
     a4paper,      % paper format
     BCOR=10mm,     % binding correction
     DIV=14,        % stripe size for margin calculation
%     liststotoc,   % table listing in toc
%     bibtotoc,     % bibliography in toc
%     idxtotoc,     % index in toc
%     parskip       % paragraph skip instad of paragraph indent
     ]{scrreprt}

%\documentclass[
%     12pt,         % font size
%     a4paper,      % paper format
%     BCOR10mm,     % binding correction
%     DIV14,        % stripe size for margin calculation
%%     liststotoc,   % table listing in toc
%%     bibtotoc,     % bibliography in toc
%%     idxtotoc,     % index in toc
%%     parskip       % paragraph skip instad of paragraph indent
%     ]{scrreprt}
%
%%%%%%%%%%%%%%%%%%%%%%%%%%%%%%%%%%%%%%%%%%%%%%%%%%%%%%%%%%%%

% PACKAGES:

% Language :
\usepackage[ngerman]{babel}
% Input and font encoding
\usepackage[utf8]{inputenc}
\usepackage[T1]{fontenc}

%% Font
%\setmainfont{Linux Libertine O}
%\setsansfont{Linux Biolinum O}

% Index-generation
\usepackage{makeidx}

% Einbinden von URLs:
\usepackage{url}
% Special \LaTex symbols (e.g. \BibTeX):
%\usepackage{doc}
% Include Graphic-files:
\usepackage{graphicx}
% Include doc++ generated tex-files:
%\usepackage{docxx}
% Include PDF links
%\usepackage[pdftex, bookmarks=true]{hyperref}

% Fuer anderthalbzeiligen Textsatz
\usepackage{setspace}

% hyperrefs in the documents
\usepackage[bookmarks=true,colorlinks,pdfpagelabels,pdfstartview = FitH,bookmarksopen = true,bookmarksnumbered = true,linkcolor = black,plainpages = false,hypertexnames = false,citecolor = black,urlcolor=black]{hyperref} 
%\usepackage{hyperref}

%%Hyphenation - Silbentrennung
%\hyphenation{Prim-ideal}

%%Bibliography
%\usepackage[backend=biber,style=alphabetic]{biblatex}
%\addbibresource{mybib.bib}
%\ExecuteBibliographyOptions{%
% isbn=false, doi=false,url=false,eprint=false%
%}%
%\DeclareSourcemap{
%  \maps[datatype=bibtex, overwrite]{
%    \map{
%      \step[fieldset=language, null]
%      \step[fieldset=note, null]
%      \step[fieldset=pagetotal, null]
%    }
%  }
%}

%Math packages
\usepackage{mathtools}%mathtools lädt amsmath
\usepackage{amssymb}
\usepackage{amsthm}

%% Use the Libertine font family for math as well
%\usepackage[libertine]{newtxmath}
%
%%Math operators
\DeclareMathOperator{\N}{\mathbb{N}}
\DeclareMathOperator{\Z}{\mathbb{Z}}
\DeclareMathOperator{\Q}{\mathbb{Q}}
\DeclareMathOperator{\R}{\mathbb{R}}
\DeclareMathOperator{\C}{\mathbb{C}}
\DeclareMathOperator{\F}{\mathbb{F}}
\DeclareMathOperator{\GL}{GL}


%%
%%PAGE REFORMATTING
%%
%\setlength{\oddsidemargin}{1.5cm}
%\setlength{\evensidemargin}{0cm}
%\setlength{\topmargin}{1mm}
%\setlength{\headheight}{1.36cm}
%\setlength{\headsep}{1.00cm}
%\setlength{\textheight}{19cm}
%\setlength{\textwidth}{14.5cm}
%\setlength{\marginparsep}{1mm}
%\setlength{\marginparwidth}{3cm}
%\setlength{\footskip}{2.36cm}

%% TODO command
%\usepackage[colorinlistoftodos,prependcaption,textsize=tiny]{todonotes}
%\newcommandx{\unsure}[2][1=]{\todo[linecolor=red,backgroundcolor=red!25,bordercolor=red,#1]{#2}}
%\newcommandx{\change}[2][1=]{\todo[linecolor=blue,backgroundcolor=blue!25,bordercolor=blue,#1]{#2}}
%\newcommandx{\info}[2][1=]{\todo[linecolor=OliveGreen,backgroundcolor=OliveGreen!25,bordercolor=OliveGreen,#1]{#2}}
%\newcommandx{\improvement}[2][1=]{\todo[linecolor=Plum,backgroundcolor=Plum!25,bordercolor=Plum,#1]{#2}}
%\newcommandx{\thiswillnotshow}[2][1=]{\todo[disable,#1]{#2}}


%%%%%%%%%%%%%%%%%%%%%%%%%%%%%%%%%%%%%%%%%%%%%%%%%%%%%%%%%%%%

% OTHER SETTINGS:

% Pagestyle:
\pagestyle{headings}

% Choose language
\newcommand{\setlang}[1]{\selectlanguage{#1}\nonfrenchspacing}


%%%%%%%%%%%%%%%%%%%%%%%%%%%%%%%%%%%%%%%%%%%%%%%%%%%%%%%%%%%%

\begin{document}
%
%   \begin{large}
%       \noindent \textbf{Notations and Nomenclature} \\
%       \vspace{1cm}
%   \end{large}
%
%
\TODO{Reminders from measure theory.  Create an introduction and motivation.}
%    We recall some definitions and results from measure theory. 
%    \bigskip
    
    %
    \noindent Let $(X, \A)$ be a measurable space and $\mu, \tilde \mu$ measures on $X$.  We say that $\mu$ is \textit{absolutely continuous} with respect to $\tilde \mu$, if
    \[\tilde \mu(A) = 0 \Rightarrow \mu(A) = 0 \quad \quad  \forAll A\in \A \]
    and denote this by $\mu \ll \tilde \mu$. \\
    %
    Let $(X, \A, \mu)$ be a measure space and $(Y, \U)$ a measurable space. For a measurable function $T: X \to Y $, 
    we denote by $T_{\#}\mu$ the \textit{pushforward measure} on $Y$ induced by $T$, i.e. the measure on $Y$ given by
    \[T_{\#}\mu(B) = \mu(T^{-1}(B)) \quad \quad \forAll  B \in \U. \]
    %
%    \noindent As $S$ is a finite set, we can write $S = \{s_1,\dots s_{|S|} \}$ and $\nu$ as a finite sum of dirac measures
%    \[\nu = \sum_{i = 1}^{|S|} {\nu_i \delta_{s_i}} \quad \text{such that} \quad \sum_{i=1}^{|S|}{\nu_i} = 1.  \]
%    %
%
%    \TODO{recall the pushforward measure}
%    \begin{verbatim}
%        https://en.wikipedia.org/wiki/Pushforward_measure
%    \end{verbatim}
%    
    \TODO{Introduction for Optimal Transport}
%    \TODO{DEFINE INTEGRABLE FUNCTIONS L }
\end{document}

%

%%%%%%%%%%%%%%%%%%%%%%%%%%%%%%%%%%%%%%%%%%%%%%%%%%%%%%%%%%%%

\chapter{Introduction}\label{Introduction}
\section{Overview/Motivation}
\section{Related Work}
\begin{itemize}
    \item Je nach dem, wie relevant die andere Dokumente sein könnten, relevante Arbeiten citieren. 
    \item Basic reference, e.g Vill, conference works
\end{itemize}
\section{Contribution/Organization of work}
\begin{itemize}
    \item Description of the approach.
    \item Distribution of this work, briefly description of the chapters.
\end{itemize}
\section{Background}
\begin{itemize}
    \item Change of variable formulas, etc.
\end{itemize}

%%%%%%%%%%%%%%%%%%%%%%%%%%%%%%%%%%%%%%%%%%%%%%%%%%%%%%%%%%%%

\chapter{Semi-dicrete Optimal Transport}\label{Optimal Transport}
\TODO{Introduction Optimal Transport}

\section{Problem}
\TODO{Introduction}
Problem von SDOT im \uline{Allgemein}

\section{Approach}
\TODO{Introduction}

\newpage
\section{Characterization of the Optimal Transport Map}
\input{Optimal_Transport/Characterization_OTM/characterization_OTM}

\newpage
\section{Basic Algorithm}
\documentclass[
     12pt,         % font size
     a4paper,      % paper format
     BCOR=10mm,     % binding correction
     DIV=14,        % stripe size for margin calculation
%     liststotoc,   % table listing in toc
%     bibtotoc,     % bibliography in toc
%     idxtotoc,     % index in toc
%     parskip       % paragraph skip instad of paragraph indent
     ]{scrreprt}

%\documentclass[
%     12pt,         % font size
%     a4paper,      % paper format
%     BCOR10mm,     % binding correction
%     DIV14,        % stripe size for margin calculation
%%     liststotoc,   % table listing in toc
%%     bibtotoc,     % bibliography in toc
%%     idxtotoc,     % index in toc
%%     parskip       % paragraph skip instad of paragraph indent
%     ]{scrreprt}
%
%%%%%%%%%%%%%%%%%%%%%%%%%%%%%%%%%%%%%%%%%%%%%%%%%%%%%%%%%%%%

% PACKAGES:

% Language :
\usepackage[ngerman]{babel}
% Input and font encoding
\usepackage[utf8]{inputenc}
\usepackage[T1]{fontenc}

%% Font
%\setmainfont{Linux Libertine O}
%\setsansfont{Linux Biolinum O}

% Index-generation
\usepackage{makeidx}

% Einbinden von URLs:
\usepackage{url}
% Special \LaTex symbols (e.g. \BibTeX):
%\usepackage{doc}
% Include Graphic-files:
\usepackage{graphicx}
% Include doc++ generated tex-files:
%\usepackage{docxx}
% Include PDF links
%\usepackage[pdftex, bookmarks=true]{hyperref}

% Fuer anderthalbzeiligen Textsatz
\usepackage{setspace}

% hyperrefs in the documents
\usepackage[bookmarks=true,colorlinks,pdfpagelabels,pdfstartview = FitH,bookmarksopen = true,bookmarksnumbered = true,linkcolor = black,plainpages = false,hypertexnames = false,citecolor = black,urlcolor=black]{hyperref} 
%\usepackage{hyperref}

%%Hyphenation - Silbentrennung
%\hyphenation{Prim-ideal}

%%Bibliography
%\usepackage[backend=biber,style=alphabetic]{biblatex}
%\addbibresource{mybib.bib}
%\ExecuteBibliographyOptions{%
% isbn=false, doi=false,url=false,eprint=false%
%}%
%\DeclareSourcemap{
%  \maps[datatype=bibtex, overwrite]{
%    \map{
%      \step[fieldset=language, null]
%      \step[fieldset=note, null]
%      \step[fieldset=pagetotal, null]
%    }
%  }
%}

%Math packages
\usepackage{mathtools}%mathtools lädt amsmath
\usepackage{amssymb}
\usepackage{amsthm}

%% Use the Libertine font family for math as well
%\usepackage[libertine]{newtxmath}
%
%%Math operators
\DeclareMathOperator{\N}{\mathbb{N}}
\DeclareMathOperator{\Z}{\mathbb{Z}}
\DeclareMathOperator{\Q}{\mathbb{Q}}
\DeclareMathOperator{\R}{\mathbb{R}}
\DeclareMathOperator{\C}{\mathbb{C}}
\DeclareMathOperator{\F}{\mathbb{F}}
\DeclareMathOperator{\GL}{GL}


%%
%%PAGE REFORMATTING
%%
%\setlength{\oddsidemargin}{1.5cm}
%\setlength{\evensidemargin}{0cm}
%\setlength{\topmargin}{1mm}
%\setlength{\headheight}{1.36cm}
%\setlength{\headsep}{1.00cm}
%\setlength{\textheight}{19cm}
%\setlength{\textwidth}{14.5cm}
%\setlength{\marginparsep}{1mm}
%\setlength{\marginparwidth}{3cm}
%\setlength{\footskip}{2.36cm}

%% TODO command
%\usepackage[colorinlistoftodos,prependcaption,textsize=tiny]{todonotes}
%\newcommandx{\unsure}[2][1=]{\todo[linecolor=red,backgroundcolor=red!25,bordercolor=red,#1]{#2}}
%\newcommandx{\change}[2][1=]{\todo[linecolor=blue,backgroundcolor=blue!25,bordercolor=blue,#1]{#2}}
%\newcommandx{\info}[2][1=]{\todo[linecolor=OliveGreen,backgroundcolor=OliveGreen!25,bordercolor=OliveGreen,#1]{#2}}
%\newcommandx{\improvement}[2][1=]{\todo[linecolor=Plum,backgroundcolor=Plum!25,bordercolor=Plum,#1]{#2}}
%\newcommandx{\thiswillnotshow}[2][1=]{\todo[disable,#1]{#2}}


%%%%%%%%%%%%%%%%%%%%%%%%%%%%%%%%%%%%%%%%%%%%%%%%%%%%%%%%%%%%

% OTHER SETTINGS:

% Pagestyle:
\pagestyle{headings}

% Choose language
\newcommand{\setlang}[1]{\selectlanguage{#1}\nonfrenchspacing}


% Avoid Problems by including other files
\usepackage{standalone}

%%%%%%%%%%%%%%%%%%%%%%%%%%%%%%%%%%%%%%%%%%%%%%%%%%%%%%%%%%%%

\begin{document}
%
%   \begin{large}
%       \noindent \textbf{Notations and Nomenclature} \\
%       \vspace{1cm}
%   \end{large}
%
%
\TODO{Something}
%    We recall some definitions and results from measure theory. 
%    \bigskip

Image = 256 x 256 x 3 \\
Over iterations nscales$=4-1 : -1 : 0$
\begin{enumerate}
    \item scale image ($2^{iter}$ ) i.e in first iter=3 : $256 / 2^3 =32$
    \item then generate on with 2 dim more, (i.e scale im with  $256 / 2^2 = 64 $)
\end{enumerate}
\begin{itemize}
    \item In first iteration, i.e iter = 3: estime adsn model on scaled image (32x32x3)
\end{itemize}

In view auf Multi-scale scaling Approach

\TODO{Something}

\end{document}


%%%%%%%%%%%%%%%%%%%%%%%%%%%%%%%%%%%%%%%%%%%%%%%%%%%%%%%%%%%%

\chapter{Multi-Layer Approach}

\section{To Define, Muli-scale?}
\begin{itemize}
    \item Decomposition of the target distribution (K-means algorithm - Clustering method)
    \item Sub/Up-sampling procedure
    \item Multi-layer model
\end{itemize}

\documentclass[
     12pt,         % font size
     a4paper,      % paper format
     BCOR=10mm,     % binding correction
     DIV=14,        % stripe size for margin calculation
%     liststotoc,   % table listing in toc
%     bibtotoc,     % bibliography in toc
%     idxtotoc,     % index in toc
%     parskip       % paragraph skip instad of paragraph indent
     ]{scrreprt}

%\documentclass[
%     12pt,         % font size
%     a4paper,      % paper format
%     BCOR10mm,     % binding correction
%     DIV14,        % stripe size for margin calculation
%%     liststotoc,   % table listing in toc
%%     bibtotoc,     % bibliography in toc
%%     idxtotoc,     % index in toc
%%     parskip       % paragraph skip instad of paragraph indent
%     ]{scrreprt}
%
%%%%%%%%%%%%%%%%%%%%%%%%%%%%%%%%%%%%%%%%%%%%%%%%%%%%%%%%%%%%

% PACKAGES:

% Language :
\usepackage[ngerman]{babel}
% Input and font encoding
\usepackage[utf8]{inputenc}
\usepackage[T1]{fontenc}

%% Font
%\setmainfont{Linux Libertine O}
%\setsansfont{Linux Biolinum O}

% Index-generation
\usepackage{makeidx}

% Einbinden von URLs:
\usepackage{url}
% Special \LaTex symbols (e.g. \BibTeX):
%\usepackage{doc}
% Include Graphic-files:
\usepackage{graphicx}
% Include doc++ generated tex-files:
%\usepackage{docxx}
% Include PDF links
%\usepackage[pdftex, bookmarks=true]{hyperref}

% Fuer anderthalbzeiligen Textsatz
\usepackage{setspace}

% hyperrefs in the documents
\usepackage[bookmarks=true,colorlinks,pdfpagelabels,pdfstartview = FitH,bookmarksopen = true,bookmarksnumbered = true,linkcolor = black,plainpages = false,hypertexnames = false,citecolor = black,urlcolor=black]{hyperref} 
%\usepackage{hyperref}

%%Hyphenation - Silbentrennung
%\hyphenation{Prim-ideal}

%%Bibliography
%\usepackage[backend=biber,style=alphabetic]{biblatex}
%\addbibresource{mybib.bib}
%\ExecuteBibliographyOptions{%
% isbn=false, doi=false,url=false,eprint=false%
%}%
%\DeclareSourcemap{
%  \maps[datatype=bibtex, overwrite]{
%    \map{
%      \step[fieldset=language, null]
%      \step[fieldset=note, null]
%      \step[fieldset=pagetotal, null]
%    }
%  }
%}

%Math packages
\usepackage{mathtools}%mathtools lädt amsmath
\usepackage{amssymb}
\usepackage{amsthm}

%% Use the Libertine font family for math as well
%\usepackage[libertine]{newtxmath}
%
%%Math operators
\DeclareMathOperator{\N}{\mathbb{N}}
\DeclareMathOperator{\Z}{\mathbb{Z}}
\DeclareMathOperator{\Q}{\mathbb{Q}}
\DeclareMathOperator{\R}{\mathbb{R}}
\DeclareMathOperator{\C}{\mathbb{C}}
\DeclareMathOperator{\F}{\mathbb{F}}
\DeclareMathOperator{\GL}{GL}


%%
%%PAGE REFORMATTING
%%
%\setlength{\oddsidemargin}{1.5cm}
%\setlength{\evensidemargin}{0cm}
%\setlength{\topmargin}{1mm}
%\setlength{\headheight}{1.36cm}
%\setlength{\headsep}{1.00cm}
%\setlength{\textheight}{19cm}
%\setlength{\textwidth}{14.5cm}
%\setlength{\marginparsep}{1mm}
%\setlength{\marginparwidth}{3cm}
%\setlength{\footskip}{2.36cm}

%% TODO command
%\usepackage[colorinlistoftodos,prependcaption,textsize=tiny]{todonotes}
%\newcommandx{\unsure}[2][1=]{\todo[linecolor=red,backgroundcolor=red!25,bordercolor=red,#1]{#2}}
%\newcommandx{\change}[2][1=]{\todo[linecolor=blue,backgroundcolor=blue!25,bordercolor=blue,#1]{#2}}
%\newcommandx{\info}[2][1=]{\todo[linecolor=OliveGreen,backgroundcolor=OliveGreen!25,bordercolor=OliveGreen,#1]{#2}}
%\newcommandx{\improvement}[2][1=]{\todo[linecolor=Plum,backgroundcolor=Plum!25,bordercolor=Plum,#1]{#2}}
%\newcommandx{\thiswillnotshow}[2][1=]{\todo[disable,#1]{#2}}


%%%%%%%%%%%%%%%%%%%%%%%%%%%%%%%%%%%%%%%%%%%%%%%%%%%%%%%%%%%%

% OTHER SETTINGS:

% Pagestyle:
\pagestyle{headings}

% Choose language
\newcommand{\setlang}[1]{\selectlanguage{#1}\nonfrenchspacing}


% Avoid Problems by including other files
\usepackage{standalone}

%%%%%%%%%%%%%%%%%%%%%%%%%%%%%%%%%%%%%%%%%%%%%%%%%%%%%%%%%%%%

\begin{document}
%
%   \begin{large}
%       \noindent \textbf{Notations and Nomenclature} \\
%       \vspace{1cm}
%   \end{large}
%
%
%\TODO{NOTES}
%    We recall some definitions and results from measure theory. 
%    \bigskip

%Image = 256 x 256 x 3 \\
%Over iterations nscales$=4-1 : -1 : 0$
%\begin{enumerate}
%    \item scale image ($2^{iter}$ ) i.e in first iter=3 : $256 / 2^3 =32$
%    \item then generate on with 2 dim more, (i.e scale im with  $256 / 2^2 = 64 $)
%\end{enumerate}
%\begin{itemize}
%    \item In first iteration, i.e iter = 3: estime adsn model on scaled image (32x32x3)
%\end{itemize}


\noindent \Large\textbf{Multi-layer approach}  \\[12pt] \normalsize
%
%\TODO{NOTES}
%
\noindent \large\textbf{Target measure decomposition}  \\[8pt] \normalsize
%
\unsure{IDEA}
\textit{Decompose} the target measure $\nu = \sum_{s\in S} \nu_s \delta_s$ at different scales using the K-means algorithm (Lloyd's algorithm) $\longrightarrow$ 
Generate a finite sequence of discrete probabity measures $\{\nu_l \}_{l=0, \dots, L}$ with decreasing support and such that $\nu_{l+1}$ should be a \textit{similar} 
to $\nu_{l}$. 
%
\unsure{MORE PRECISELY}
Using a clustering algorithm, generate a finite sequence of finite sets $\{ S \equal S^0, \dots, S^L\}$ (to use as support for the discrete probabilty distributions) such that $|S^l| < |S^{l+1}|$ for all $l\in\{0,\dots,L-1\} $ (and $|S^L|=1$).
Then, define $\nu^0 \coloneqq \nu$ and use successively transport maps to define the distributions on the supports $S^l$, i.e. define measurable maps 
\[\pi_l : S^l \to S^{l+1} \quad   \text{and set}\quad \nu^{l+1} \coloneqq {\pi_l}_{\#}\nu^l. \]
Thus, we get get successibly %for $l = 0, \dots, L-1 $ %
probabilty measures $\nu^{l+1} = \sum_{s\in S^{l+1}} \nu^{l+1}_s \delta_s$ supported on $S^{l+1}$ satisfying 
%
\[ \nu_s^{l+1} = \nu^{l+1}(s) = \nu^l(\pi_l^{-1}(s)) = \sum_{p \in \pi_l^{-1}(s) } \nu^l_p .\]
%
\textbf{Idea: } Using Lloyd's algorithm we get: $\pi_l:  x \mapsto \argmin_{s\in S^{l+1}} \| x - s \|^2 .$ \\[12pt]
%
\noindent \large\textbf{Multi-layer Transport Map - With 2 Layers (as in Paper)}  \\[8pt] \normalsize
%\textbf{?Main idea here?!COPYPASTE:} Use multi-scale representation of the target distribution to sequentially estimate a hierarchical Laguerre cell partitioning of the source distribution \\
\textbf{Reminder from last chapter (is written different):}
\begin{align*}
    h^\nu (x,W) : \R^d \times \R^{|S|} &\to \R, \\
    (x,W) &\mapsto \min_{s\in S} \pow_W(x,s) + \langle W, \nu \rangle =  (\min_{s\in S} \|x-s\|^2 -W(s)) + \langle W, \nu \rangle
\end{align*}
Then we have 
\[ \nabla_W h^\nu = \nu -\1_{T_W(x)=s}^{S}. \]
Where 
\[\1_{T_W(x)=s}^{S}: S \to \R, \quad     
            x \mapsto
            \begin{cases} 
                1 &\mbox{if } x=T_W(x) \\
                0 & \mbox{else } 
            \end{cases} . \]

\textbf{Sketch of the algorithm:}\\[8pt]
\underline{Given:} $\mu$ (Target distribution), $\nu$ (Source distribution), $L=2$ (number of layers). 
\begin{itemize}
    \item Decompose target measure: $\{\nu^l\}_{l=0,1}$, $\{S^l\}_{l=0,1}$ as above.
%    \item Set $\omega^l = \nu^l$,  for $l=0,1$.
    \item Set $W^l = 0$,  for $l=0,1$. (Weights to be computed)
    \item Set $n^l: S^l \to 0 $,  for $l=0,1$. (Number of visits of points in $S^l$)
\end{itemize}
\underline{Apply ASGD.}  At each iteration: sample $x \sim \mu$ and then:
\begin{enumerate}
    \item (L=1: first layer) Compute
        \[\tilde s = \argmin_{s \in S^{1}} \| x - s \|^2 - W^1(s) .\]
        I.o.w. compute $T_{W^1}(x)$. If $W^1=0$ (as in the first iterations), this is equivalent to computing a Least-squares. 
    \item Compute gradient 
        \[g = \nabla_{W^1} h^{\nu^1} = \nu^1 - \1_{s=\tilde s}^{S^1} \]
        Where \[\1_{s=\tilde s}^{S^1}: S^1 \to \R, \quad     
            x \mapsto
            \begin{cases} 
                1 &\mbox{if } x=\tilde s \\
                0 & \mbox{else } 
            \end{cases} . \]
    \item Update $W^1$ as in Algorithm 1: 
        \[W^1 \longleftarrow \text{Use gradient, gradient-step, iteration } (g, C, iter) \]
    \item Update number of visits 
        \[n^1(\tilde s) = n^1(\tilde s) + 1 \]
    \item (L=0: second layer) Compute
        \[\tilde {\tilde{s}} = \argmin_{s \in \pi_0^{-1}(\tilde s)} \| x - s \|^2 - W^0(s) \]
        I.o.w. compute $T_{W^0|_{\pi_0^{-1}(\tilde s)}}|^{\pi_0^{-1}(\tilde s)}(x) = T_{W^0}|^{\pi_0^{-1}(\tilde s)}(x)$, 
        where $T_{W^0}|^{\pi_0^{-1}(\tilde s)}$ denotes the map $T_{W^0}$ 
        with restricted codomain ${\pi_0^{-1}(\tilde s)}$. \\ 
        Observations: 
        \begin{itemize}
            \item This computations are faster than computing
                \[T_{W^0}(x) = \argmin_{s \in S^0} \| x - s \|^2 - W^0(s), \]
                as $|\pi_0^{-1}(\tilde s)| < |S^0|$.
            \item It may happend (yes? when?) that
                \[T_{W^0}(x) \in S^0 \setminus \pi_0^{-1}(\tilde s). \]
                Consequences?
        \end{itemize}
    \item Compute gradient 
        \[\tilde g = \nabla_{W^0|_{\pi^{-1}(\tilde s)}} h^{\nu^0|_{\pi^{-1}(\tilde s)}} = \nu^0|_{\pi^{-1}(\tilde s)} - \1_{s=\tilde{\tilde{s}}}^{\pi^{-1}(\tilde s)} \]
        Where \[\1_{s=\tilde{\tilde{s}}}^{\pi^{-1}(\tilde s)}: \pi^{-1}(\tilde s) \to \R, \quad     
            x \mapsto
            \begin{cases} 
                1 &\mbox{if } x=\tilde{\tilde{s}} \\
                0 & \mbox{else } 
            \end{cases}  \]
    \item Update $W^0$ as in Algorithm 2: 
        \[W^0 \longleftarrow \text{Use gradient, gradient-step, number of visits} (\tilde g, C, n^1)  \]
        Actually, only update entries on $\pi^{-1}(\tilde s)$, i.e 
        \[ W^0|_{\pi^{-1}(s)} \longleftarrow \text{Use gradient, gradient-step, number of visits} (\tilde g, C, n^1)  \]
\end{enumerate}
%        \begin{enumerate}
%            \item[(i)] 
%                Compute 
%        \end{enumerate}


\end{document}


\section{Algorithm (ASGD)} 
%%%%%%%%%%%%%%%%%%%%%%%%%%%%%%%%%%%%%%%%%%%%%%%%%%%%%%%%%%%%


\chapter{Application: Texture Synthesis}
%\TODO{Introduction}
%\section{First Level of Synthesis}
\section{Patch distributions}
%\TODO{Introduction}
\subsection{Source Distribution}
\subsubsection{Gaussian Random Fields (Gaussian Synthesis)}
\subsubsection{Gaussian Mixture Models}
\subsubsection{Gaussian patches distribution and properties}

%\begin{itemize}
%    \item Gaussian Random Fields (Gaussian Synthesis)
%    \item Gaussian mixture models
%    \item Gaussian patches distribution
%    \item Properties
%\end{itemize}

\subsection{Target Distribution}
\subsubsection{Emprirical patch distribution and simplifications}

%\begin{itemize}
%    \item Emprirical patch distribution
%\end{itemize}

\section{Optimal Transport in Patch Space}
\subsection{Local Transformations (Patch convolutions)}

%\begin{itemize}
%    \item Local transformations (patch convolutions) on the Gaussian Synthesis
%\end{itemize}

%%%%%%%%%%%%%%%%%%%%%%%%%%%%%%%%%%%%%%%%%%%%%%%%%%%%%%%%%%%%
%\chapter{Stochastic Optimization}
%\TODO{Introduction}
%\section{Average Stochastic Gradient Descent}
%\subsection{ASGD for the estimation of the multi-layer map}
%\subsubsection{Sampling on the source distribution}
%%%%%%%%%%%%%%%%%%%%%%%%%%%%%%%%%%%%%%%%%%%%%%%%%%%%%%%%%%%%
%\chapter{\sout{Numerical Experiments}}
%%%%%%%%%%%%%%%%%%%%%%%%%%%%%%%%%%%%%%%%%%%%%%%%%%%%%%%%%%%%
\chapter{Conclusion}

%%%%%%%%%%%%%%%%%%%%%%%%%%%%%%%%%%%%%%%%%%%%%%%%%%%%%%%%%%%%
%\label{chap:main}
%%%%%%%%%%%%%%%%%%%%%%%%%%%%%%%%%%%%%%%%%%%%%%%%%%%%%%%%%%%%%
%\newpage
%
%\chapter{Mein Beitrag}
%\label{chap:main}
%
%Dieses Kapitel stellt meist den Hauptteil der Arbeit dar. Vor dem
%ersten Abschnitt sollte ein kurzer berblick (ein paar wenige Sze
%mit Verweise auf nachfolgende Abschnitte) gegeben werden. Beispiel: Im
%nachfolgenden Abschnitt \ref{sec:overview} wir ein berblick ber die Anforderungen an
%das Modell gegeben. 
%
%\section{berblick und Zielsetzung}
%\label{sec:overview} 
%
%Knapp zwei Seiten, in dem die Anforderungen, die Zielsetzung und die
%Methoden berblicksartig beschrieben werden. Hier sollte die
%Beschreibung ``technischer'' bzw.~``formaler'' sein als in der
%Einleitung, da der Leser nun mit den Grundlagen und verwandten
%Arbeiten vertraut ist.
%
%\section{Erster Teil}
%\label{sec:teil1}
%
%In diesem und den nachfolgenden Abschnitten werden die Beitrge der
%Arbeit motiviert, formal sauber (oft mathematisch, sprich mit
%Definitionen etc.) beschrieben, und bei Bedarf mithilfe von Beispielen
%verdeutlicht. Die Beschreibungen in diesem Kapitel sind meist
%unabhngig von einer konkreten Realisierung und Daten; diese werden im
%nachfolgenden Kapitel detailliert.
%
%\section{Zweiter Teil}
%\label{sec:teil2}
%
%Usw.
%
%%%%%%%%%%%%%%%%%%%%%%%%%%%%%%%%%%%%%%%%%%%%%%%%%%%%%%%%%%%%%
%\chapter{Experimentelle Evaluation}
%\label{chap:eval}
%
%Der Aufbau dieses Kapitels oder dessen Aufteilung in zwei Kapiteln ist
%stark von dem Thema und der Bearbeitung des Themas
%abhngig. Beschrieben werden hier Daten, die fr eine Evaluation
%verwendet wird (Quellen, Beispiele, Statistiken), die Zielsetzung der Evaluation
%und die verwendeten Mae sowie die Ergebnisse (u.a.~mithilfe von
%Charts, Diagrammen, Abbildungen etc.)
%
%Dieses Kapitel kann auch mit einer Beschreibung der Realisierung eines
%Systems beginnen (kein Quellcode, maximal Klassendiagramme!).
%
%%%%%%%%%%%%%%%%%%%%%%%%%%%%%%%%%%%%%%%%%%%%%%%%%%%%%%%%%%%%%
%\chapter{Zusammenfassung und Ausblick}
%\label{chap:concl}
%
%Hier werden noch einmal die wichtigsten Ergebnisse und Erkenntnisse
%der Arbeit zusammengefasst (nicht einfach eine Wiederholung des
%Aufbaus der vorherigen Kapitel!), welche neuen Konzepte, Methoden und
%Werkzeuge Neues entwickelt wurden, welche Probleme nun (effizienter)
%gelst werden knnen, und es wird ein Ausblick auf weiterfhrende
%Arbeiten gegeben (z.B.~was Sie machen wrden, wenn Sie noch 6 Monate
%mehr Zeit htten).
%%%%%%%%%%%%%%%%%%%%%%%%%%%%%%%%%%%%%%%%%%%%%%%%%%%%%%%%%%%%



%%%%%%%%%%%%%%%%%%%%%%%%%%%%%%%%%%%%%%%%%%%%%%%%%%%%%%%%%%%%
% References (Literaturverzeichnis):
% a) Style (with abbreviations: use alpha):
% see
% https://de.wikibooks.org/wiki/LaTeX-W%C3%B6rterbuch:_bibliographystyle
% for the different formats and styles
%\cite*{Agg07}

\cite{*}
\bibliographystyle{apalike}
% b) The File:
\bibliography{references}

\end{document}
