\documentclass[
     12pt,         % font size
     a4paper,      % paper format
     BCOR=10mm,     % binding correction
     DIV=14,        % stripe size for margin calculation
%     liststotoc,   % table listing in toc
%     bibtotoc,     % bibliography in toc
%     idxtotoc,     % index in toc
%     parskip       % paragraph skip instad of paragraph indent
     ]{scrreprt}

%\documentclass[
%     12pt,         % font size
%     a4paper,      % paper format
%     BCOR10mm,     % binding correction
%     DIV14,        % stripe size for margin calculation
%%     liststotoc,   % table listing in toc
%%     bibtotoc,     % bibliography in toc
%%     idxtotoc,     % index in toc
%%     parskip       % paragraph skip instad of paragraph indent
%     ]{scrreprt}
%
%%%%%%%%%%%%%%%%%%%%%%%%%%%%%%%%%%%%%%%%%%%%%%%%%%%%%%%%%%%%

% PACKAGES:

% Language :
\usepackage[ngerman]{babel}
% Input and font encoding
\usepackage[utf8]{inputenc}
\usepackage[T1]{fontenc}

%% Font
%\setmainfont{Linux Libertine O}
%\setsansfont{Linux Biolinum O}

% Index-generation
\usepackage{makeidx}

% Einbinden von URLs:
\usepackage{url}
% Special \LaTex symbols (e.g. \BibTeX):
%\usepackage{doc}
% Include Graphic-files:
\usepackage{graphicx}
% Include doc++ generated tex-files:
%\usepackage{docxx}
% Include PDF links
%\usepackage[pdftex, bookmarks=true]{hyperref}

% Fuer anderthalbzeiligen Textsatz
\usepackage{setspace}

% hyperrefs in the documents
\usepackage[bookmarks=true,colorlinks,pdfpagelabels,pdfstartview = FitH,bookmarksopen = true,bookmarksnumbered = true,linkcolor = black,plainpages = false,hypertexnames = false,citecolor = black,urlcolor=black]{hyperref} 
%\usepackage{hyperref}

%%Hyphenation - Silbentrennung
%\hyphenation{Prim-ideal}

%%Bibliography
%\usepackage[backend=biber,style=alphabetic]{biblatex}
%\addbibresource{mybib.bib}
%\ExecuteBibliographyOptions{%
% isbn=false, doi=false,url=false,eprint=false%
%}%
%\DeclareSourcemap{
%  \maps[datatype=bibtex, overwrite]{
%    \map{
%      \step[fieldset=language, null]
%      \step[fieldset=note, null]
%      \step[fieldset=pagetotal, null]
%    }
%  }
%}

%Math packages
\usepackage{mathtools}%mathtools lädt amsmath
\usepackage{amssymb}
\usepackage{amsthm}

%% Use the Libertine font family for math as well
%\usepackage[libertine]{newtxmath}
%
%%Math operators
\DeclareMathOperator{\N}{\mathbb{N}}
\DeclareMathOperator{\Z}{\mathbb{Z}}
\DeclareMathOperator{\Q}{\mathbb{Q}}
\DeclareMathOperator{\R}{\mathbb{R}}
\DeclareMathOperator{\C}{\mathbb{C}}
\DeclareMathOperator{\F}{\mathbb{F}}
\DeclareMathOperator{\GL}{GL}


%%
%%PAGE REFORMATTING
%%
%\setlength{\oddsidemargin}{1.5cm}
%\setlength{\evensidemargin}{0cm}
%\setlength{\topmargin}{1mm}
%\setlength{\headheight}{1.36cm}
%\setlength{\headsep}{1.00cm}
%\setlength{\textheight}{19cm}
%\setlength{\textwidth}{14.5cm}
%\setlength{\marginparsep}{1mm}
%\setlength{\marginparwidth}{3cm}
%\setlength{\footskip}{2.36cm}

%% TODO command
%\usepackage[colorinlistoftodos,prependcaption,textsize=tiny]{todonotes}
%\newcommandx{\unsure}[2][1=]{\todo[linecolor=red,backgroundcolor=red!25,bordercolor=red,#1]{#2}}
%\newcommandx{\change}[2][1=]{\todo[linecolor=blue,backgroundcolor=blue!25,bordercolor=blue,#1]{#2}}
%\newcommandx{\info}[2][1=]{\todo[linecolor=OliveGreen,backgroundcolor=OliveGreen!25,bordercolor=OliveGreen,#1]{#2}}
%\newcommandx{\improvement}[2][1=]{\todo[linecolor=Plum,backgroundcolor=Plum!25,bordercolor=Plum,#1]{#2}}
%\newcommandx{\thiswillnotshow}[2][1=]{\todo[disable,#1]{#2}}


%%%%%%%%%%%%%%%%%%%%%%%%%%%%%%%%%%%%%%%%%%%%%%%%%%%%%%%%%%%%

% OTHER SETTINGS:

% Pagestyle:
\pagestyle{headings}

% Choose language
\newcommand{\setlang}[1]{\selectlanguage{#1}\nonfrenchspacing}


% Avoid Problems by including other files
\usepackage{standalone}

%%%%%%%%%%%%%%%%%%%%%%%%%%%%%%%%%%%%%%%%%%%%%%%%%%%%%%%%%%%%

\begin{document}
    %
    \documentclass[
     12pt,         % font size
     a4paper,      % paper format
     BCOR=10mm,     % binding correction
     DIV=14,        % stripe size for margin calculation
%     liststotoc,   % table listing in toc
%     bibtotoc,     % bibliography in toc
%     idxtotoc,     % index in toc
%     parskip       % paragraph skip instad of paragraph indent
     ]{scrreprt}

%\documentclass[
%     12pt,         % font size
%     a4paper,      % paper format
%     BCOR10mm,     % binding correction
%     DIV14,        % stripe size for margin calculation
%%     liststotoc,   % table listing in toc
%%     bibtotoc,     % bibliography in toc
%%     idxtotoc,     % index in toc
%%     parskip       % paragraph skip instad of paragraph indent
%     ]{scrreprt}
%
%%%%%%%%%%%%%%%%%%%%%%%%%%%%%%%%%%%%%%%%%%%%%%%%%%%%%%%%%%%%

% PACKAGES:

% Language :
\usepackage[ngerman]{babel}
% Input and font encoding
\usepackage[utf8]{inputenc}
\usepackage[T1]{fontenc}

%% Font
%\setmainfont{Linux Libertine O}
%\setsansfont{Linux Biolinum O}

% Index-generation
\usepackage{makeidx}

% Einbinden von URLs:
\usepackage{url}
% Special \LaTex symbols (e.g. \BibTeX):
%\usepackage{doc}
% Include Graphic-files:
\usepackage{graphicx}
% Include doc++ generated tex-files:
%\usepackage{docxx}
% Include PDF links
%\usepackage[pdftex, bookmarks=true]{hyperref}

% Fuer anderthalbzeiligen Textsatz
\usepackage{setspace}

% hyperrefs in the documents
\usepackage[bookmarks=true,colorlinks,pdfpagelabels,pdfstartview = FitH,bookmarksopen = true,bookmarksnumbered = true,linkcolor = black,plainpages = false,hypertexnames = false,citecolor = black,urlcolor=black]{hyperref} 
%\usepackage{hyperref}

%%Hyphenation - Silbentrennung
%\hyphenation{Prim-ideal}

%%Bibliography
%\usepackage[backend=biber,style=alphabetic]{biblatex}
%\addbibresource{mybib.bib}
%\ExecuteBibliographyOptions{%
% isbn=false, doi=false,url=false,eprint=false%
%}%
%\DeclareSourcemap{
%  \maps[datatype=bibtex, overwrite]{
%    \map{
%      \step[fieldset=language, null]
%      \step[fieldset=note, null]
%      \step[fieldset=pagetotal, null]
%    }
%  }
%}

%Math packages
\usepackage{mathtools}%mathtools lädt amsmath
\usepackage{amssymb}
\usepackage{amsthm}

%% Use the Libertine font family for math as well
%\usepackage[libertine]{newtxmath}
%
%%Math operators
\DeclareMathOperator{\N}{\mathbb{N}}
\DeclareMathOperator{\Z}{\mathbb{Z}}
\DeclareMathOperator{\Q}{\mathbb{Q}}
\DeclareMathOperator{\R}{\mathbb{R}}
\DeclareMathOperator{\C}{\mathbb{C}}
\DeclareMathOperator{\F}{\mathbb{F}}
\DeclareMathOperator{\GL}{GL}


%%
%%PAGE REFORMATTING
%%
%\setlength{\oddsidemargin}{1.5cm}
%\setlength{\evensidemargin}{0cm}
%\setlength{\topmargin}{1mm}
%\setlength{\headheight}{1.36cm}
%\setlength{\headsep}{1.00cm}
%\setlength{\textheight}{19cm}
%\setlength{\textwidth}{14.5cm}
%\setlength{\marginparsep}{1mm}
%\setlength{\marginparwidth}{3cm}
%\setlength{\footskip}{2.36cm}

%% TODO command
%\usepackage[colorinlistoftodos,prependcaption,textsize=tiny]{todonotes}
%\newcommandx{\unsure}[2][1=]{\todo[linecolor=red,backgroundcolor=red!25,bordercolor=red,#1]{#2}}
%\newcommandx{\change}[2][1=]{\todo[linecolor=blue,backgroundcolor=blue!25,bordercolor=blue,#1]{#2}}
%\newcommandx{\info}[2][1=]{\todo[linecolor=OliveGreen,backgroundcolor=OliveGreen!25,bordercolor=OliveGreen,#1]{#2}}
%\newcommandx{\improvement}[2][1=]{\todo[linecolor=Plum,backgroundcolor=Plum!25,bordercolor=Plum,#1]{#2}}
%\newcommandx{\thiswillnotshow}[2][1=]{\todo[disable,#1]{#2}}


%%%%%%%%%%%%%%%%%%%%%%%%%%%%%%%%%%%%%%%%%%%%%%%%%%%%%%%%%%%%

% OTHER SETTINGS:

% Pagestyle:
\pagestyle{headings}

% Choose language
\newcommand{\setlang}[1]{\selectlanguage{#1}\nonfrenchspacing}


%%%%%%%%%%%%%%%%%%%%%%%%%%%%%%%%%%%%%%%%%%%%%%%%%%%%%%%%%%%%

\begin{document}
   \begin{Large}
       \noindent \textbf{ Nomenclature }
       \hspace{2cm}
   \end{Large}

   \begin{flalign*}
        &\| \cdot \|&                       &\text{Euclidean Norm on } \R^d&\\
        &\langle \cdot, \cdot \rangle&      &\text{Scalar product on } \R^d&\\
        &\R_{\ge 0}&                        &\text{Positive real numbers}&\\
        &\lambda^d&                         &\text{Lebesgue measure on} \R^d&\\
        &\B^d &                             &\text{Borel} \sigma\text{-algebra on } \R^d&\\
        &\Pot(\cdot)&                       &\text{Power set of a set}&\\
        &|\cdot|&                           &\text{Cardinality of a set}&\\
        &\LL(\mu)&                          &\mu \text{-integrable functions}&\\
    \end{flalign*}


\end{document}
 
    %\newpage
    % 
    \documentclass[
     12pt,         % font size
     a4paper,      % paper format
     BCOR=10mm,     % binding correction
     DIV=14,        % stripe size for margin calculation
%     liststotoc,   % table listing in toc
%     bibtotoc,     % bibliography in toc
%     idxtotoc,     % index in toc
%     parskip       % paragraph skip instad of paragraph indent
     ]{scrreprt}

%\documentclass[
%     12pt,         % font size
%     a4paper,      % paper format
%     BCOR10mm,     % binding correction
%     DIV14,        % stripe size for margin calculation
%%     liststotoc,   % table listing in toc
%%     bibtotoc,     % bibliography in toc
%%     idxtotoc,     % index in toc
%%     parskip       % paragraph skip instad of paragraph indent
%     ]{scrreprt}
%
%%%%%%%%%%%%%%%%%%%%%%%%%%%%%%%%%%%%%%%%%%%%%%%%%%%%%%%%%%%%

% PACKAGES:

% Language :
\usepackage[ngerman]{babel}
% Input and font encoding
\usepackage[utf8]{inputenc}
\usepackage[T1]{fontenc}

%% Font
%\setmainfont{Linux Libertine O}
%\setsansfont{Linux Biolinum O}

% Index-generation
\usepackage{makeidx}

% Einbinden von URLs:
\usepackage{url}
% Special \LaTex symbols (e.g. \BibTeX):
%\usepackage{doc}
% Include Graphic-files:
\usepackage{graphicx}
% Include doc++ generated tex-files:
%\usepackage{docxx}
% Include PDF links
%\usepackage[pdftex, bookmarks=true]{hyperref}

% Fuer anderthalbzeiligen Textsatz
\usepackage{setspace}

% hyperrefs in the documents
\usepackage[bookmarks=true,colorlinks,pdfpagelabels,pdfstartview = FitH,bookmarksopen = true,bookmarksnumbered = true,linkcolor = black,plainpages = false,hypertexnames = false,citecolor = black,urlcolor=black]{hyperref} 
%\usepackage{hyperref}

%%Hyphenation - Silbentrennung
%\hyphenation{Prim-ideal}

%%Bibliography
%\usepackage[backend=biber,style=alphabetic]{biblatex}
%\addbibresource{mybib.bib}
%\ExecuteBibliographyOptions{%
% isbn=false, doi=false,url=false,eprint=false%
%}%
%\DeclareSourcemap{
%  \maps[datatype=bibtex, overwrite]{
%    \map{
%      \step[fieldset=language, null]
%      \step[fieldset=note, null]
%      \step[fieldset=pagetotal, null]
%    }
%  }
%}

%Math packages
\usepackage{mathtools}%mathtools lädt amsmath
\usepackage{amssymb}
\usepackage{amsthm}

%% Use the Libertine font family for math as well
%\usepackage[libertine]{newtxmath}
%
%%Math operators
\DeclareMathOperator{\N}{\mathbb{N}}
\DeclareMathOperator{\Z}{\mathbb{Z}}
\DeclareMathOperator{\Q}{\mathbb{Q}}
\DeclareMathOperator{\R}{\mathbb{R}}
\DeclareMathOperator{\C}{\mathbb{C}}
\DeclareMathOperator{\F}{\mathbb{F}}
\DeclareMathOperator{\GL}{GL}


%%
%%PAGE REFORMATTING
%%
%\setlength{\oddsidemargin}{1.5cm}
%\setlength{\evensidemargin}{0cm}
%\setlength{\topmargin}{1mm}
%\setlength{\headheight}{1.36cm}
%\setlength{\headsep}{1.00cm}
%\setlength{\textheight}{19cm}
%\setlength{\textwidth}{14.5cm}
%\setlength{\marginparsep}{1mm}
%\setlength{\marginparwidth}{3cm}
%\setlength{\footskip}{2.36cm}

%% TODO command
%\usepackage[colorinlistoftodos,prependcaption,textsize=tiny]{todonotes}
%\newcommandx{\unsure}[2][1=]{\todo[linecolor=red,backgroundcolor=red!25,bordercolor=red,#1]{#2}}
%\newcommandx{\change}[2][1=]{\todo[linecolor=blue,backgroundcolor=blue!25,bordercolor=blue,#1]{#2}}
%\newcommandx{\info}[2][1=]{\todo[linecolor=OliveGreen,backgroundcolor=OliveGreen!25,bordercolor=OliveGreen,#1]{#2}}
%\newcommandx{\improvement}[2][1=]{\todo[linecolor=Plum,backgroundcolor=Plum!25,bordercolor=Plum,#1]{#2}}
%\newcommandx{\thiswillnotshow}[2][1=]{\todo[disable,#1]{#2}}


%%%%%%%%%%%%%%%%%%%%%%%%%%%%%%%%%%%%%%%%%%%%%%%%%%%%%%%%%%%%

% OTHER SETTINGS:

% Pagestyle:
\pagestyle{headings}

% Choose language
\newcommand{\setlang}[1]{\selectlanguage{#1}\nonfrenchspacing}


%%%%%%%%%%%%%%%%%%%%%%%%%%%%%%%%%%%%%%%%%%%%%%%%%%%%%%%%%%%%

\begin{document}
%
%   \begin{large}
%       \noindent \textbf{Notations and Nomenclature} \\
%       \vspace{1cm}
%   \end{large}
%
%
\TODO{Motivation for Optimal Transport \& recall some definitions from measure theory}
%    We recall some definitions and results from measure theory. 
%    \bigskip
    
    %
    \noindent Let $(X, \A)$ be a measurable space and $\mu, \tilde \mu$ two measures on it.  We say that $\mu$ is absolutetly continuos with respect to $\tilde \mu$ and write $\mu \ll \tilde \mu$, if
    %if every set with measure zero with respect to $\tilde \mu$ has measure zero with respect to $\mu$. This means 
    \[\tilde \mu(A) = 0 \Rightarrow \mu(A) = 0 \quad \quad  \forall A\in \A. \]
    %
    Let $(X, \A, \mu)$ be a measure space and $(Y, \U)$ a measurable space. For a measurable function $T: X \to Y $, 
    we denote by $T_{\#}\mu$ the pushforward measure on $Y$ induced by $T$, i.e
    \[T_{\#}\mu(B) = \mu(T^{-1}(B)) \quad \quad \forall B \in \U. \]
    %
%    \noindent As $S$ is a finite set, we can write $S = \{s_1,\dots s_{|S|} \}$ and $\nu$ as a finite sum of dirac measures
%    \[\nu = \sum_{i = 1}^{|S|} {\nu_i \delta_{s_i}} \quad \text{such that} \quad \sum_{i=1}^{|S|}{\nu_i} = 1.  \]
%    %
%
%    \TODO{recall the pushforward measure}
%    \begin{verbatim}
%        https://en.wikipedia.org/wiki/Pushforward_measure
%    \end{verbatim}
%    
    \TODO{}
%    \TODO{DEFINE INTEGRABLE FUNCTIONS L }
\end{document}

    %%

    %%
    
    For a given finite set $S \subset \R^d$, we want to study the following problem: \\ 
    Given two probilty spaces $(\R^d$, $\B^d$, $\mu)$ and $(S, \Pot(S), \nu)$, we want to minimize 
    %
    \begin{align} \label{eq::main}
        \integral {\R^d} {\|x-T(x) \|^2} {\mu}  
    \end{align}
    %
    over all measurable maps $T: \R^d \to S$, that satisfy $T_{\#}\mu = \nu$. \\%[8pt]
    %
    Recall that as $(S, \Pot(S), \nu)$ is a finite probability space, we can write $\nu$ as a finite sum of dirac measures
    \begin{align} \label{eq::dirac} 
        \nu = \sum_{s\in S} {\nu_s \delta_{s}} \quad \text{where   } \nu_s \in \R_{\ge 0} \text{   and such that} \quad \sum_{s \in S}{\nu_s} = 1. 
    \end{align}
    %
    Thus, the problem we are considering consists in finding a measurable map $T$ that minimizes the functional (\ref{eq::main}) and that fullfills 
    $\mu (T^{-1}(s)) = \nu_s$ for all $s\in S$. For $s\in S$, we will call $\mu(T^{-1}(s)) $ the capacity of $s$.
    %
    \TODO{** Prove existence and uniqueness of a solution}
    %
    \indent As shown by [AHA], finding a measurable map that minimizes the functional (\ref{eq::main}) is equivalent to finding the maximum of a concave function and thus, can be solved with
    standard optimization methods. The formulation of this optimization problem is based in two steps.
    First, we find a minimizer of the functional (\ref{eq::main}) over all measurable functions with same capacities.  This optimal solution $T_W$ is inspired geometrically and constructed using 
    a predefine \textit{weight vector} $W$.\\
    %This weights are defined for every point in $S$ and have a geometrical interpretation, which we will describe below. 
    The next step will consist in adapting this weight vector, sucht that the condition $T_{W_{\#}}\mu = \nu$ is fullfilled.\\ 
    The motivation behind this approach is inspired geometrically by studying a generalization of \textit{Voroni diagrams}. 
    For clarity, we recall the definition of this diagrams and review the necessary concepts needed for this approach. 
    %
    \begin{defi}[Voronoi Diagrams]
        Let $S\subset \R^d$ be a finite set.  We define for every point $s\in S$
        \[\reg(s) \coloneqq \{x \in \R^d : \|x-s\| \le \|x-\tilde s \| \quad  \forall \tilde s \in S\setminus \{s\} \}. \]
        We call this, the region (or cell) of the point $s$.
        The partition of $\R^d$ created by the union of the regions of all points is called the Voronoi diagram of $S$.
    \end{defi}
    %
    %
    %
    \begin{rem}
        Note that the partition of $\R^d$ generated by the Voronoi diagram is given by convex regions.  Such a partition induces naturally a map $T: \R^d \to S$, which assign each point in $\R^d$ the corresponding 
        point in $S$ of the cell where it is located, i.e
         \begin{align} \label{eq::reg}
             T(x) = s \quad \Leftrightarrow \quad x\in \reg(s).  
         \end{align}
         By defintion, some points in $\R^d$ may belong to more than one region.  By convention, $T$ assigns those points an arbitrary one in $S$ of a region
         where it is located. We call $T$, the from the Voronoi diagram induced assigment. 
    \end{rem}
    %
    %
    %
    A generalization of the presented concepts arise when using another distance function for the definition of the regions. 
    %This allow us to vary the induced assignment.
    One practial way for this, appears by using the \textit{power function with weights $W$.}

    \begin{defi}[Power function]      
        Let $S\subset \R^d$ be a finite set and $W: S \to \R $ a function on $S$. The power function with weights $W$ is defined as
        \[ \pow_W(x,s) = \|x-s\|^2 - W(s). \]
        W is called weight function on S.
    \end{defi}
    %
    %
    %
    \begin{rem}
        For simplicity of notation we will indentify sometimes the weight function $W:S\to\R^d$  as a vector in $\R^{|S|}$.  We call then
        $W$, weight vector on S.
    \end{rem}
    %
    %
    Similarly as by Voronoi Diagrams, we can define regions on $\R^d$ by using the power function with weights $W$.
    For a point $s\in S$ we call 
    \[\reg_W(s) \coloneqq \{x \in \R^d : \pow_W(x,s) \le \pow_W(x,\tilde s) \quad  \forall \tilde s \in S\setminus \{s\} \} \]
    the \textit{power region} (or power cell) of $s$ with weights $W$. Power regions also create a partition of $\R^d$ which is called the \textit{power diagram} of $S$ with weights $W$. \\
    The geometric intution behind the definition of power diagrams appears by looking the spheres around $s\in S$ with positive radius %$\sqrt {W(s)}$
    %
    \[\Sph_{\sqrt W}^{d-1}(s) \coloneqq \{x \in \R^d : \| x-s \| = \sqrt{W(s)} \} \]
    %
    when $W : S \to \R_{>0} $. The power function $\pow(\cdot, s)$ for a fixed $s\in S$, returns a negative (resp. positive) value whenever $x\in \R^d$ is inside 
    (resp. outside) the sphere $\Sph_{\sqrt W}^{d-1}(s)$ and zero when $s\in \Sph_{\sqrt W}^{d-1}(s)$. Thus, increasing (resp. decreasing) the values of the weights $W(s)$ on each point $s$ would expand (resp. schrink)
    the power cells. 
    %
    %
    \begin{rem}
        Unlike Voronoi diagrams, the power cells of a point $s\in S$ may not contain the point $s$ or may be even empty. Nevertheless, the power diagram still partioned $\R^d$ in 
        convex polyhedron.
    \end{rem}
    %
    %
    By replacing $\reg$ with $\reg_W$ in (\ref{eq::reg}) we obtain a map $T_W: \R^d \to S$ which depends on $W$.  Similarly as by Voronoi diagrams, we assign those points who share
    different cells, an arbitrarly point $s\in S$ of those shared regions. We call this map, the \textit{power assignment} of $S$ with weights $W$. \\
    %
    Power assignments have a natural optimization property, since by definition it holds 
    %an optimization property, which we recall in the following Lemma.
    \begin{align}
        ( T_W(x) = s \  \Leftrightarrow \ x\in \reg_W(s) ) \quad \Leftrightarrow \quad T_W(x) = \min_{s\in S} \|x-s\|^2 - W(s) \label{eq::minimalityPower}
    \end{align}
    %
    for all points $x \in \R^d$ which doesn't share different regions. In fact, power functions even minimize the functional (\ref{eq::main}) for a fixed predifined weight vector $W$.
    We will prove this in Lemma \ref{lemma::1step}. Consequently, the next natural question will deal the choice of the weights $W$, s.t it fullfills the condition $T_{W_{\#}}\mu = \nu$.\\[8pt]
    %
    \indent We recall the change of variables theorem from the measure theory. 
    %
    \begin{thm*}[change of variables]
         Let $(X, \A, \mu)$ be a measure space, $(Y, \U)$ a measurable space and $T: X \to Y $ a measurable function.
         For a measurable function $f : Y \to \R^d $ holds $f\in \LL(T_\#\mu) \Leftrightarrow f \circ T \in \LL(\mu) $ and when one of this statments is true then 
         %
         \begin{align} \label{eq::transformel}
             \integral {T^{-1}(B)} {f} {T_{\#}\mu} = \integral {B} {f \circ T} {\mu} \quad \text{for all } B\in \U.
         \end{align}
         %
    \end{thm*}
    % 
    \begin{proof}[Proof]
    Measure theory, e.g p.$191$ [J.E]
    \end{proof}
    % Seite 191 Measure theory
    
%    %
%    $x \mapsto \argmin_{s\in S} \|x-s\|. $  Note that some points   
%
%    all measurable functions
%    $T: \R^d \to S$ satisfying $\mu(T^{-1}(s)) = w_s$, where $w_s \in \R$ are fixed values defined for every point $s$ in $S$. 
%    \TODO{ ....}
%
%
%    \TODO{ formulate differnt this lemma}
    %
    \begin{rem*} \label{rem::measurabilty}
        Note that as the power region of a point $s\in S\subset \R^d$ is even an empty set or a convex polyhedra, it is measurable with respect to the Lebesgue measure $\lambda^d$ on $\R^d$.
        Denoting by $\mathring B$ the interior of a set $B$ with respect to the standard topology, it holds 
        %
        \[\lambda^d(\reg(s)) = \lambda^d(\mathring{\reg(s)}).\]
        %
        For a Probabilty space $(\R^d,\B^d,\mu)$ s.t $\mu \ll \lambda^d$, holds then
        \[\mu(\reg(s)) = \mu(\mathring{\reg(s)}) \quad \quad \text{and} \quad \quad \sum_{s\in S} \mu(\reg(s))=1.\]

        
        %exists by the Radon-Nikodym a non-negative function $f\in \LL(\mu)$, s.t 
        %\[\mu (B) = \integral {B} {f} {\mu} \quad \quad \forall B\in \B^d. \]
        %%
        %Thus, it holds 
    \end{rem*}
    %
    %
    \begin{lem} \label{lemma::1step}
        Let $(\R^d,\B,\mu)$ be a Probabilty space, s.t $\mu \ll \lambda^d$.  Let $S$ be a finite subset of $\R^d$ with weights $W$ and $\zeta: S \to \R_{\ge 0}$ be a function on S.
        Then, the power assignment $T_W $ minimizes 
        %\[\integral {R^D} {\rho(x) \|x-T(x) \|} {\lambda^d} \]
        \[\integral {\R^d} {\|x-T(x) \|^2} {\mu} \]
        over all measurable maps $T : \R^d \to S $ with capacities $\mu(T^{-1} (s)) = \zeta(s)  $ for all $s\in S$.
    \end{lem}
    %
    \begin{proof}[Proof]
        Using the minimality condition (\ref{eq::minimalityPower}) of power assignments, it holds \[\pow_W(x,T_W(x)) \le \pow_W(x,s) \] for all $s\in S$. 
        Consequently, $T_W$ minimizes 
        \[ \integral {\R^d} {\pow_W(x,T(x))} {\mu} =  \integral{\R^d} {\| x - T(x) \|^2} {\mu} - \integral{\R^d} {\omega(T(x))} {\mu} \]
%        \begin{align*} 
%            \integral {\R^D} {\pow_W(x,T(x))} {\mu} &=  \integral{\R^d} {\| x - T(x) \|^2} {\mu} - \integral{\R^d} {\omega(T(x))} {\mu}\\
%                                                    &\overset {\text{explain}}=  \integral{\R^d} {\rho(x)\| x - T(x) \|^2} {\lambda^d} - \integral{\R^d} {\omega(T(x))} {\mu}
%        \end{align*}
        over all measurable maps $T:\R^d \to S$. Using the fact that $\R^d = \bigcup_{s\in S} \reg(s)$ in combination with remark \ref{rem::measurabilty}, it holds
        %and that the boundaries of the power cells have zero $\lambda^d$-measure and thus also zero $\mu$-measure, it holds
        \begin{align*} 
            %\integral{\R^d} {\omega(T(x))} {\mu} &\overset {(\ref{eq::transformel})}=  \sum_{s\in S} \integral{s} {\omega(T(x))} {\mu^T} 
            %                                     = \sum_{s\in S} \integral{s} {\omega(T(x))} {\mu^T} \\
            %                                     &= \sum_{s\in S} \mu(T^{-1}(s))\omega(s) \text{ \ which is constant}
            \integral{\R^d} {\omega(T(x))} {\mu} &= \sum_{s\in S} \integral{\reg(s)} {W(T(x))} {\mu} \\ 
                                                 &\overset {(\ref{eq::transformel})}= \sum_{s\in S} \integral{s} {W} {T_{\#}\mu} \\
                                                 &= \sum_{s\in S} \mu(T^{-1}(s))W(s) \\
                                                 &= \sum_{s\in S} \zeta(s)W(s) 
        \end{align*}
        which is constant for a fixed $\zeta$ and $W$.
    \end{proof}

    The natural question to handle next, is how to chose $W$, s.t the condition $T_{W_{\#}}\mu = \nu$ holds. As we will show below, this question can be equivalently formulated
    as finding the maximum of a concave function. In order to achive this, we first recall the original setting of our originial problem and introduce some definitions. \\[8pt]
    %
    \indent Let $(\R^d, \B, \mu)$ and $(S, \Pot(S), \nu)$ be two proabilty spaces s.t $\mu \ll \lambda^d$. As in (\ref{eq::dirac}), we write  $\nu = \sum_{s\in S} {\nu_s \delta_{s}} $ as 
    a finite sum of dirac measures. \\
    For $\FF \coloneqq \{ f: \R^d \to S : f \text{ is measurable}\}$, define
    %
    \[L: \FF \times \R^{|S|} \to \R, \quad (T,W) \mapsto \integral {\R^d} {\pow_W(x,T(x))} {\mu}. \]
    %
    This map has important properties, which we will show and then transfer them to the concave function of our reformulated problem. For a map $T\in \FF$, let 
    %
    \[\zeta_T : S \to \R, \quad s\mapsto \mu(T^{-1}(s)) \]
    be the vector of capacities induced by $T$ and 
    %
    \[Q: \FF \to \R, \quad T \mapsto \integral {\R^d} {\|x-T(x)\|^2} {\mu} \]
    %
    be the functional that we want to study. As shown in Lemma \ref{lemma::1step}, it holds
    %
    \[L(T,W) = Q(T) - \langle \zeta_T, W \rangle. \]
    %
    And hence, $L_T \coloneqq L(T, \cdot)$ defines a linear function on $\R^{|S|}$ for a fixed $T \in \FF$. \\
    Recall that for a given $W \in \R^{|S|}$ and a power assignment $T_W$, holds \\ 
    $\pow_W(x,T_W(x)) \le \pow_W(x,s)$ for all $s \in S$. Consequently, for a fixed $W \in \R^d$ holds 
    %
    %
    \begin{align} \label{eq:propTW}
        T_W(W) = \argmin_{T\in \FF} L(T,W).
    \end{align}
    %
    %
    We claim that \[ f : \R^{|S|} \to \R^d, \quad W \mapsto L(T_W, W) = L_{T_W}(W) \]
    is concave.
    %\TODO{Rewrite properties for the proof}
    \begin{proof}[Proof]
        Let $\alpha \in [0,1]$ and $W_1, W_2 \in \R^{|S|}$, then
        \begin{align*}
            f(\alpha W_1 + (1- \alpha)W_2) &= L_{T_{\alpha W_1 + (1- \alpha)W_2}}(\alpha W_1 + (1- \alpha)W_2) \\
                                        &= L_{T_{\alpha W_1 + (1- \alpha)W_2}}(\alpha W_1) + L_{T_{\alpha W_1 + (1- \alpha)W_2}}((1-\alpha) W_2)\\
                                        &\overset {\text{(\ref{eq:propTW})}}\ge L_{T_{\alpha W_1}}(\alpha W_1) + L_{T_{(1-\alpha) W_2}}((1- \alpha) W_2) \\
                                        &= \alpha L_{T_{\alpha W_1}}(W_1) + (1-\alpha) L_{T_{(1-\alpha) W_2}}(W_2) \\ 
                                        &\overset {\text{(\ref{eq:propTW})}}\ge \alpha L_{T_{W_1}}(W_1) + (1-\alpha) L_{T_{W_2}}(W_2) = \alpha f(W_1) + (1-\alpha) f(W_2)
        \end{align*}
    \end{proof}
    \TODO{Prove formally that f is smooth}

    $f$ is smooth and the gradient of $f$ at $W$ is given by $\nabla f(W) = - \zeta_{T_{W}}$.
    Recall that because of Lemma \ref{lemma::1step}, to solve our problem (\ref{eq::main}) we need to find a weight vector $W^*$ s.t $T_{{W^*}_{\#}}(\mu) = \nu$. 
    In other words, it should hold $\mu(T_{W^*}^{-1} (s)) = \nu_s $ for all $s\in S$. \\
    Consider now the function
    %
    \[H: \R^{|S|} \to \R, \quad W \mapsto f(W) + \langle \nu, W \rangle = \langle \nu - \zeta_{T_{W}}, W \rangle + Q(T_W). \]
    %
    This function is concave and differentiable as a sum of concave differentiable functions, it holds $\nabla H(W) = \nu - \zeta_{T_{W}}$ and hence
    \[ T_{W_{\#}}\mu(s) =\mu(T_W^{-1}(s)) = \nu_s \quad \forall s\in S \quad  \Leftrightarrow \quad  \zeta_{T_{W}} = \nu \quad  \Leftrightarrow \quad \nabla H(W) = 0. \]
    %
    Thus finding a solution of the problem (\ref{eq::main}) is equivalent to finding a maximum of the concave function $H$. For this function holds
    \begin{align*}
        \frac{\partial H}{W(s)}  &=  \frac{\partial f(W)}{W(s)}  + \frac{\partial \langle \nu, W \rangle}{W(s)} \\
                                &= -\mu(T_W^{-1}(s)) + W(s) \\
                                &= -\mu(\reg(s)) + W(s).
    \end{align*}
    
    This optimization problem has a probabilistic interpration. If we realize $X$ as a random variable of distribution $\mu$, i.e $X \sim \mu$. Then, defining
    %
    \[h_W^\nu(x) \coloneqq \min_{s\in S} \|x-s\|^2 - W(s) + \langle W, \nu  \rangle = \|x-T_W(x)\|^2 -W(s) + \langle W, \nu  \rangle, \]
    %
    it holds 
    %\[h_W(x, s) = \min_{s\in S} \|x-s\|^2 - W(s) + \langle W, \nu  \rangle \]
    \begin{align*}
        \E[h_W^\nu(X)] &= \integral{\R^d} {\min_{s\in S} \|X-s\|^2 - W(s)} {\mu} + \integral{\R^d} {\langle W, \nu  \rangle } {\mu} \\
                    &= \integral{\R^d} {\|x-T_W(X)\|^2 -W(s)} {\mu} + \langle W, \nu  \rangle  = H(W).
    \end{align*}

    Thus, our problem consists in minimizing the expectation of $h_W^\nu(X)$.
     
    
\end{document}

