% PACKAGES:

% Language :
\usepackage[ngerman]{babel}
% Input and font encoding
\usepackage[utf8]{inputenc}
\usepackage[T1]{fontenc}

%% Font
%\setmainfont{Linux Libertine O}
%\setsansfont{Linux Biolinum O}

% Index-generation
\usepackage{makeidx}

% Einbinden von URLs:
\usepackage{url}
% Special \LaTex symbols (e.g. \BibTeX):
%\usepackage{doc}
% Include Graphic-files:
\usepackage{graphicx}
% Include doc++ generated tex-files:
%\usepackage{docxx}
% Include PDF links
%\usepackage[pdftex, bookmarks=true]{hyperref}

% Fuer anderthalbzeiligen Textsatz
\usepackage{setspace}

% hyperrefs in the documents
\usepackage[bookmarks=true,colorlinks,pdfpagelabels,pdfstartview = FitH,bookmarksopen = true,bookmarksnumbered = true,linkcolor = black,plainpages = false,hypertexnames = false,citecolor = black,urlcolor=black]{hyperref} 
%\usepackage{hyperref}

%%Hyphenation - Silbentrennung
%\hyphenation{Prim-ideal}

%%Bibliography
%\usepackage[backend=biber,style=alphabetic]{biblatex}
%\addbibresource{mybib.bib}
%\ExecuteBibliographyOptions{%
% isbn=false, doi=false,url=false,eprint=false%
%}%
%\DeclareSourcemap{
%  \maps[datatype=bibtex, overwrite]{
%    \map{
%      \step[fieldset=language, null]
%      \step[fieldset=note, null]
%      \step[fieldset=pagetotal, null]
%    }
%  }
%}

%Math packages
\usepackage{mathtools}%mathtools lädt amsmath
\usepackage{amssymb}
\usepackage{amsthm}

%% Use the Libertine font family for math as well
%\usepackage[libertine]{newtxmath}
%
%%Math operators
\DeclareMathOperator{\A}{\mathcal{A}}
\DeclareMathOperator{\B}{\mathcal{B}}
\DeclareMathOperator{\C}{\mathbb{C}}
\DeclareMathOperator{\E}{\mathbb{E}}
\DeclareMathOperator{\U}{\mathcal{U}}
\DeclareMathOperator{\N}{\mathbb{N}}
\DeclareMathOperator{\Z}{\mathbb{Z}}
\DeclareMathOperator{\Q}{\mathbb{Q}}
\DeclareMathOperator{\R}{\mathbb{R}}
\DeclareMathOperator{\FF}{\mathcal{F}}
\DeclareMathOperator{\Sph}{\mathbb{S}}
\DeclareMathOperator{\F}{\mathbb{F}}
\DeclareMathOperator{\K}{\mathbb{K}}
\DeclareMathOperator{\LL}{\mathcal{L}}
\DeclareMathOperator{\dd}{{\mathrm{d}}}
\DeclareMathOperator{\argmax}{arg\,max}
\DeclareMathOperator{\argmin}{arg\,min}
\DeclareMathOperator{\GL}{GL}
\DeclareMathOperator{\RE}{Re}
\DeclareMathOperator{\IM}{Im}
\DeclareMathOperator{\pow}{pow}
\DeclareMathOperator{\reg}{reg}
\DeclareMathOperator{\Pot}{Pot}

%%Define some variables
\newcommand{\integral}[3]{\int_{#1} #2 \ {\dd} #3 }  
%\newcommand{\f}[2]{\sum_{n=#1}^{#2}\frac{\mu(n) \ \log(n) }{n} }	% SI LA USO
%\newcommand{\MM}[2]{\sum_{n=#1}^{#2}\mu(n)}
%\newcommand{\marker}[1]{\textbf{\textcolor{red}{(#1)}}}
%\newcommand{\Mob}{$\mathfrak{M}$}
%\newcommand{\Aut}{$Aut(\overline {\mathbb C})$}
%\newcommand{\definicion}{\coloneqq}	

%Theoremstyle  https://de.sharelatex.com/learn/Theorems_and_proofs#Reference_guide			

\theoremstyle{plain}
\newtheorem{thm}{Theorem}[chapter] % reset theorem numbering for each chapter
\newtheorem{cor}[thm]{Corollary} % definition numbers are dependent on theorem numbers
\newtheorem{lem}[thm]{Lemma}
\newtheorem{lemm}[thm]{Lemma}
\newtheorem{def+thm}[thm]{Definition and Theorem} 
%
\theoremstyle{plain}
\newtheorem*{thm*}{Theorem}
%
\theoremstyle{remark}
\newtheorem*{rem}{Remark}
%
\theoremstyle{remark}
\newtheorem{rem*}[thm]{Remark}
%
\theoremstyle{definition}
\newtheorem{defi}[thm]{Definition} % definition numbers are dependent on theorem numbers
\newtheorem{exa}[thm]{Example} % same for example numbers

%%
%%PAGE REFORMATTING
%%
%\setlength{\oddsidemargin}{1.5cm}
%\setlength{\evensidemargin}{0cm}
%\setlength{\topmargin}{1mm}
%\setlength{\headheight}{1.36cm}
%\setlength{\headsep}{1.00cm}
%\setlength{\textheight}{19cm}
%\setlength{\textwidth}{14.5cm}
%\setlength{\marginparsep}{1mm}
%\setlength{\marginparwidth}{3cm}
%\setlength{\footskip}{2.36cm}

%% TODO command
\usepackage[colorinlistoftodos,prependcaption,textsize=tiny]{todonotes}
\newcommand{\TODO}[1]{\todo[backgroundcolor=red!25,bordercolor=red,inline]{TODO: \ \ #1}}
%\newcommandx{\unsure}[2][1=]{\todo[linecolor=red,backgroundcolor=red!25,bordercolor=red,#1]{#2}}
%\newcommandx{\change}[2][1=]{\todo[linecolor=blue,backgroundcolor=blue!25,bordercolor=blue,#1]{#2}}
%\newcommandx{\info}[2][1=]{\todo[linecolor=OliveGreen,backgroundcolor=OliveGreen!25,bordercolor=OliveGreen,#1]{#2}}
%\newcommandx{\thiswillnotshow}[2][1=]{\todo[disable,#1]{#2}}


%%%%%%%%%%%%%%%%%%%%%%%%%%%%%%%%%%%%%%%%%%%%%%%%%%%%%%%%%%%%

% OTHER SETTINGS:

% Pagestyle:
\pagestyle{headings}

% Choose language
\newcommand{\setlang}[1]{\selectlanguage{#1}\nonfrenchspacing}
