% !TeX root = main.tex

% LaTeX-Template by Lukas Sauer.
% To compile correctly including the bibliography, compile once using LuaLaTex. Compile again using biber. At the end, compile again using LuaLaTex.


\documentclass[abstracton,titlepage=on,toc=bib]{scrartcl}


%Typesetting packages
\usepackage[english,ngerman]{babel}
\usepackage{fontspec}
\usepackage[babel,german=quotes]{csquotes}
\usepackage{multicol}

\setmainfont{Linux Libertine O}
\setsansfont{Linux Biolinum O}


%Hyphenation - Silbentrennung
\hyphenation{Prim-ideal}

%Bibliography
\usepackage[backend=biber,style=alphabetic]{biblatex}
\addbibresource{mybib.bib}
\ExecuteBibliographyOptions{%
 isbn=false, doi=false,url=false,eprint=false%
}%
\DeclareSourcemap{
  \maps[datatype=bibtex, overwrite]{
    \map{
      \step[fieldset=language, null]
      \step[fieldset=note, null]
      \step[fieldset=pagetotal, null]
    }
  }
}


%Math packages
\usepackage{mathtools}%mathtools lädt amsmath
\usepackage{amssymb}
\usepackage{amsthm}
% Use the Libertine font family for math as well
\usepackage[libertine]{newtxmath}








%Math operators
\DeclareMathOperator{\N}{\mathbb{N}}
\DeclareMathOperator{\Z}{\mathbb{Z}}
\DeclareMathOperator{\Q}{\mathbb{Q}}
\DeclareMathOperator{\R}{\mathbb{R}}
\DeclareMathOperator{\C}{\mathbb{C}}
\DeclareMathOperator{\F}{\mathbb{F}}
\DeclareMathOperator{\GL}{GL}




%Document Commands
\NewDocumentCommand\defn{m}{\emph{#1}}



\usepackage[colorlinks=true,linkcolor=black,citecolor=black,filecolor=black,urlcolor=black]{hyperref}
\usepackage[capitalize,nameinlink,ngerman]{cleveref}
%Theorem Environments and their formats
\theoremstyle{plain}
\newtheorem{stz}{Satz}[section]
\newtheorem{thm}[stz]{Theorem}
\newtheorem{lem}[stz]{Lemma}
\newtheorem{folg}[stz]{Folgerung}
\newtheorem{cor}[stz]{Korollar}
\newtheorem*{stz*}{Satz}
\theoremstyle{definition}
\newtheorem{defin}[stz]{Definition}
%Hyperref options
\providecommand*{\lemautorefname}{Lemma}
\providecommand*{\stzautorefname}{Satz}
\providecommand*{\thmautorefname}{Theorem}
\providecommand*{\folgautorefname}{Folgerung}
\providecommand*{\corautorefname}{Korollar}

\let\geq\geqslant
\let\leq\leqslant

% YOU PROBABLY DON'T NEED THIS.
\usepackage{blindtext}

\begin{document}
%%Titel
\titlehead{Ruprecht-Karls-Universität Heidelberg\\ Fakultät für Mathematik und Informatik\\ Mathematisches Institut}
\subject{Bachelorarbeit}
\title{Titel der Bachelorarbeit}
%\subtitle{Untertitel }
\author{Maria Musterfrau}
\date{eingereicht am 30. Februar 2128}
\publishers{betreut von Prof. Dr. Jane Doe}




\maketitle


\KOMAoptions{titlepage=false}%Damit beide Abstracts auf der gleichen Seite sind.
\begin{addmargin*}{1,8cm}\topskip0pt
\vspace*{\fill}
Hiermit erkläre ich, Maria Musterfrau, geb. am 01.01.1999 in Berlin, dass ich die vorliegende Bachelorarbeit selbst verfasst habe und keine anderen als die angegeben Quellen und Hilfsmittel verwendet habe.\\~\\~\\
Heidelberg, den 6. Oktober 2016\hfill\makebox[4cm]{\hrulefill}
\vspace*{\fill}
\end{addmargin*}\break



% !TeX root = main.tex

\begin{abstract}
Diese Zusammenfassung der Abschlussarbeit ist auf Deutsch und wird korrekt getrennt.

\blindtext[1]
\end{abstract}
\begin{otherlanguage}{english}
\begin{abstract}
This abstract of the thesis is in English and the syllables are divided correctly.

\blindtext[1]
\end{abstract}
\end{otherlanguage}
\break


\tableofcontents
\break
\addsec{Einleitung}
% !TeX root = ../main.tex

\enquote{Als Gregor Samsa eines Morgens aus unruhigen Träumen erwachte, fand er sich in seinem Bett zu einem ungeheueren Ungeziefer verwandelt. Er lag auf seinem panzerartig harten Rücken und sah, wenn er den Kopf ein wenig hob, seinen gewölbten, braunen, von bogenförmigen Versteifungen geteilten Bauch, auf dessen Höhe sich die Bettdecke, zum gänzlichen Niedergleiten bereit, kaum noch erhalten konnte. Seine vielen, im Vergleich zu seinem sonstigen Umfang kläglich dünnen Beine flimmerten ihm hilflos vor den Augen.} Mit diesen Worten beginnt nicht nur diese Arbeit, sondern auch Franz Kafkas berühmtes Werk \enquote{Die Verwandlung} \cite[vgl.][]{kafka1915}.
\blindtext[2]
\pagebreak
\section{Grundlegende Aussagen über Ringe}
% !TeX root = ../main.tex

\subsection{Eigenschaften von $\Z$}



\begin{defin}
Ein \defn{Ring} $R$ ist eine algebraische Struktur, die bestimmte Eigenschaften erfüllt.
\end{defin}

\begin{stz}
Jeder abelsche Gruppe ist ein $\Z$-Modul.
\end{stz}
\begin{proof}\cite[vgl.][Prop. 0.0.0]{lang2005}
Klar.
\end{proof}

Die normalen Zeichen für die Relation $3\geq 2$ habe ich durch dieses schöneren Zeichen ersetzt. Die mathematischen Zeichen $\Z$, $\C$, $\R$ etc. kann man mit \texttt{\textbackslash Z}, \texttt{\textbackslash C} etc. im Mathemodus setzen. Matrixgruppen lassen sich z.B. so schreiben: $\GL_n(\R)$.

\blindtext[6]

\break
\printbibliography

\end{document}
