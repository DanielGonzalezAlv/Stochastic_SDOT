% !TeX root = ../main.tex

\subsection{Eigenschaften von $\Z$}



\begin{defin}
Ein \defn{Ring} $R$ ist eine algebraische Struktur, die bestimmte Eigenschaften erfüllt.
\end{defin}

\begin{stz}
Jeder abelsche Gruppe ist ein $\Z$-Modul.
\end{stz}
\begin{proof}\cite[vgl.][Prop. 0.0.0]{lang2005}
Klar.
\end{proof}

Die normalen Zeichen für die Relation $3\geq 2$ habe ich durch dieses schöneren Zeichen ersetzt. Die mathematischen Zeichen $\Z$, $\C$, $\R$ etc. kann man mit \texttt{\textbackslash Z}, \texttt{\textbackslash C} etc. im Mathemodus setzen. Matrixgruppen lassen sich z.B. so schreiben: $\GL_n(\R)$.

\blindtext[6]